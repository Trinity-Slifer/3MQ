\documentclass[12pt,a4paper]{report}
\usepackage{graphicx}
\usepackage[margin=1in]{geometry}
\usepackage{faktor}
\usepackage{amsmath}
\usepackage[makeroom]{cancel}
\usepackage{mathtools}
\usepackage{physics}
\usepackage{siunitx}
\usepackage{amsthm}
\usepackage{dynkin-diagrams}
\usepackage{wrapfig}
\usepackage{subcaption}
\usepackage{amssymb}
\usepackage{dsfont}
\usepackage{float}
\usepackage{booktabs}
\usepackage{multirow}
\usepackage[T1]{fontenc}
\usepackage[utf8]{inputenc}
\usepackage[italian]{babel}
\usepackage{newlfont}
\usepackage{color}

\theoremstyle{definition}
\newtheorem{definition}{Definizione}[section]
\newtheorem{example}[definition]{Esempio}
\newtheorem{observation}[definition]{Osservazione}
\numberwithin{equation}{section}
\theoremstyle{plain}
\newtheorem{theorem}[definition]{Teorema}
\newtheorem{proposition}[definition]{Proposizione}
\newtheorem{corollary}[definition]{Corollario}
\newtheorem{lemma}[definition]{Lemma}
\theoremstyle{remark}
\newtheorem{remark}[definition]{Remark}
\newtheorem{postulato}[definition]{Postulato}
\input xy
\xyoption{all}


\DeclareMathOperator\Sym{Sym}
\DeclareMathOperator\End{End}
\DeclareMathOperator\Aut{Aut}
\DeclareMathOperator\Der{Der}
\DeclareMathOperator\ad{ad}
\DeclareMathOperator\Ker{Ker}
\DeclareMathOperator\Hom{Hom}
\DeclareMathOperator\gr{gr}
\DeclareMathOperator\mult{mult}
\DeclareMathOperator\Sw{ S(\mathbb{R}^n) }
\DeclareMathOperator\Cc{ C^\infty_c(\mathbb{R}^n) }
\DeclareMathOperator\Lr{ L^2(\mathbb{R}^n) }
\DeclareMathOperator\B{ \mathfrak{B}(\mathcal{H}) }
\DeclareMathOperator\Ct{ \mathfrak{B}_1(\mathcal{H}) }
\DeclareMathOperator\Hs{ \mathfrak{B}_2(\mathcal{H}) }
\DeclareMathOperator\Bo{ \mathfrak{B}_\infty(\mathcal{H}) }
\DeclareMathOperator\dt{ \mathcal{D}(T)}
\DeclareMathOperator\hi{\mathcal{H}}
\DeclareMathOperator\Ht{\hat{\otimes}}
\DeclareMathOperator\un{\mathds{1}}



\begin{document}


\title{Dispense 3MQ}
\date{Gennaio 2025}
\maketitle
\vspace{8.0cm}

Le dimostrazioni segnate con * non sono state svolte a lezione e alle volte richiederanno teoremi e metodi non visti, a meno che non sia specificato diversamente è possibile trovare tutto in \cite{Mor}.
$\B \; \Cc \Sw$
\vspace{1.0cm}


\newpage
\tableofcontents 
\newpage


\include{Chap3}
\include{Chap2}
\chapter{Monaco}


Chiameremo $\Bo$ l'insieme degli operatori compatti su $\hi$.

\begin{definition}
    Sia $\hi$ con la norma indotta $\| \|$ dal prodotto scalare, diremo che $A \in \B$ è un operatore di Hilber-Schmidt se esiste una base hilbertiana $\{ u_k\}$ tale che $\sum \|Au_k \|<\infty$. 
\end{definition}

Indicheremo la classe di operatori di Hilber-Schmidt su $\hi$ come $\Hs$, è possibile inoltre dimostrare che 





\begin{definition}
    Una mappa lineare $\Phi : \B \to \mathbb{C}$ è detta \emph{ una famiglia di valori attesi} se valgono le seguenti:

\begin{itemize}
    \item $\Phi(\un) = 1$.
    \item $\Phi(A)$ è reale quando $A$ è autoaggiunto.
    \item $\Phi(A)$ è positivo quando $A$ è autoaggiunto e positivo.
    \item Per ogni successione $A_n \in \B$ se $\| A_n\psi - A \psi\| \to 0$ per tutti $\psi \in \hi$ allora $\Phi(A_n) \to \Phi(A)$.
\end{itemize}
\end{definition}

\begin{definition}
    Un operatore $\rho\in \Ct$ è una matrice densità se $\rho$ è autoaggiunto, non negativo e vale $\Tr \rho =1$.
\end{definition}

\begin{definition}
    Siano $\hi_1, \hi_2$ spazi di Hilber definiremo il \emph{prodotto tensore tra due spazi di Hilbert} $\hi_1 \hat{\otimes} \hi_2$ come il completamento di $\hi_1 \otimes \hi_2$ rispetto al prodotto:
$(u_1 \otimes v_1, u_2 \otimes v_2)= (u_1, u_2)_1 (v_1, v_2)_2$.
\end{definition}



\begin{proposition}
    L'applicazione $ L^2(X_1, \mu_1) \Ht L^2(X_2, \mu_2) \to L^2(X_1 \times X_2, \mu_1 \times \mu_2) $ è un isomorfismo.
\end{proposition}

Per la dimostrazione di veda \cite{Hall}.

Se ho n sistemi quantistici composti descritti da $\hi_1, \dots, \hi_n$ allora il sistema composto è descritto dallo spazio di hilber $\hi = \hi_1 \Ht \dots \Ht \hi_n$. 
Per cui un sistema quantistico di n particelle puó essere descritto da $L^2(\mathbb{R}^{nd})= \Lr \Ht \dots \Ht \Lr$. Ora vogliamo generalizzare gli operatori a questi spazi tensore, 
siano $A_i^* = A_i\in \mathfrak{B}(\hi_1)$ definiremo l'osservabile del sistema composto come $A(\psi_1 \otimes \dots \otimes \psi_n)= A_1 \psi_1 \otimes \dots \otimes A_n \psi_n$.
In particolare:
\begin{enumerate}
    \item Se $A_i^* = A_i\in \mathfrak{B}(\hi_i)$ definisco $\B \ni A^{ (i)} := \un \otimes \un \otimes \dots \otimes \un \otimes A_i \otimes \un \otimes \dots \otimes \un$.
    \item Se $I = \{ i_1, \dots , i_k \} \subset \{1, \dots , n \}$ e $A \in \mathfrak{B}(\hi_{i_1}\Ht \dots \Ht \hi_{i_k})$ allora $A^{ (I)} \in \B$ agisce sui sottosistemi associati.
\end{enumerate}

\begin{theorem}
    Se $\rho^* = \rho \geq 0$ $\Tr(\rho) = 1$ matrice densità su $\hi$ allora siste una matrice densità $\rho^(I)$ su $\mathfrak{B}(\hi_{i_1}\Ht \dots \Ht \hi_{i_k})$ tale che 
\end{theorem}







\begin{thebibliography}{20}

\bibitem{Mor}
V. Moretti (2012) \emph{Teoria spettrale e meccanica quantistica}, Springer.

\bibitem{Hall}
Brian C. Hall (2013) \emph{Quantum Theory for Mathematicians}, Springer New York. 

\end{thebibliography}




\end{document}


