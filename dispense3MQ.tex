\documentclass[12pt,a4paper]{report}
\usepackage{graphicx}
\usepackage[margin=1in]{geometry}
\usepackage{faktor}
\usepackage{amsmath}
\usepackage[makeroom]{cancel}
\usepackage{mathtools}
\usepackage{physics}
\usepackage{siunitx}
\usepackage{amsthm}
\usepackage{dynkin-diagrams}
\usepackage{wrapfig}
\usepackage{subcaption}
\usepackage{amssymb}
\usepackage{dsfont}
\usepackage{float}
\usepackage{booktabs}
\usepackage{multirow}
\usepackage[T1]{fontenc}
\usepackage[utf8]{inputenc}
\usepackage[italian]{babel}
\usepackage{newlfont}
\usepackage{color}

\theoremstyle{definition}
\newtheorem{definition}{Definizione}[section]
\newtheorem{example}[definition]{Esempio}
\newtheorem{observation}[definition]{Osservazione}
\numberwithin{equation}{section}
\theoremstyle{plain}
\newtheorem{theorem}[definition]{Teorema}
\newtheorem{proposition}[definition]{Proposizione}
\newtheorem{corollary}[definition]{Corollario}
\newtheorem{lemma}[definition]{Lemma}
\theoremstyle{remark}
\newtheorem{remark}[definition]{Remark}
\newtheorem{postulato}[definition]{Postulato}
\input xy
\xyoption{all}


\DeclareMathOperator\Sym{Sym}
\DeclareMathOperator\End{End}
\DeclareMathOperator\Aut{Aut}
\DeclareMathOperator\Der{Der}
\DeclareMathOperator\ad{ad}
\DeclareMathOperator\Ker{Ker}
\DeclareMathOperator\Ran{Ran}
\DeclareMathOperator\Hom{Hom}
\DeclareMathOperator\mult{mult}
\DeclareMathOperator\Sw{ \mathcal{S}(\mathbb{R}^d) }
\DeclareMathOperator\mom{ \hat{\mathbb{R}^d} }
\DeclareMathOperator\Cc{ C^\infty_c(\mathbb{R}^d) }
\DeclareMathOperator\Lr{ L^2(\mathbb{R}^d) }
\DeclareMathOperator\R{\mathbb{R}}
\DeclareMathOperator\B{ \mathfrak{B}(\mathcal{H}) }
\DeclareMathOperator\Ct{ \mathfrak{B}_1(\mathcal{H}) }
\DeclareMathOperator\Hs{ \mathfrak{B}_2(\mathcal{H}) }
\DeclareMathOperator\Bo{ \mathfrak{B}_\infty(\mathcal{H}) }
\DeclareMathOperator\dt{ \mathcal{D}(T)}
\DeclareMathOperator\ex{ \mathcal{E}}
\DeclareMathOperator\hi{\mathcal{H}}
\DeclareMathOperator\Ht{\hat{\otimes}}
\DeclareMathOperator\un{\mathds{1}}



\begin{document}


\title{Dispense 3MQ}
\date{Gennaio 2025}
\maketitle
\vspace{8.0cm}

Le dimostrazioni segnate con * non sono state svolte a lezione e alle volte richiederanno teoremi e metodi non visti, a meno che non sia specificato diversamente è possibile trovare tutto in \cite{Mor}.
$\B \; \Cc \Sw$
\vspace{1.0cm}


\newpage
\tableofcontents 
\newpage

\section{La struttura teorica della meccanica quantistica}


Per poter creare una teoria matematica della meccanica quantistica dobbiamo prima interpretare la cosa che in fisica è la piú importante, la struttura sperimentale.

\begin{figure}[H]
\begin{center}
 \begin{tikzpicture}[scale = 1]

\draw [fill=orange!20] (-8,0) rectangle (8,-4);
\draw [fill=orange!20] (-8,-6) rectangle (8,-10);
\draw [fill=orange!20] (-8,-12) rectangle (8,-16);

\draw[orange, thick] (-8,0) to (8,0);
\draw[orange, thick] (-8,-4) to (8,-4);
\draw[orange, thick] (-8,0) to (-8,-4);
\draw[orange, thick] (8,0) to (8,-4);
\draw[orange, thick] (-8,-6) to (8,-6);
\draw[orange, thick] (-8,-10) to (8,-10);
\draw[orange, thick] (-8,-6) to (-8,-10);
\draw[orange, thick] (8,-6) to (8,-10);
\draw[orange, thick] (-8,-12) to (8,-12);
\draw[orange, thick] (-8,-16) to (8,-16);
\draw[orange, thick] (-8,-12) to (-8,-16);
\draw[orange, thick] (8,-12) to (8,-16);


 \end{tikzpicture}
 \end{center}
\end{figure}

Di tutte le possibili misurazioni per prima cosa ci ridurremo a quelle cheriguardano \emph{proposizioni elementari }che definirimemo come le proposizioni esprimili nel sistema fisico che nella singola ripetizione possono essere o vere o false. Per esempio in meccanica classica data una particella in $\mathbb{R}^3$ una proposizione elementare puó essere costruita per ogni $\Omega \in \mathbb{R}^3$ come $e_\Omega := \text{" la particella è vincolata in $\Omega$"}$. Per semplicità inizieremo a studiare lo schema in figura prima dal punto di vista classico e poi proveremo a crearne una versione coerente con l'odierna interpretazione della meccanica quantistica. 

In meccanica classica, nello specifico in meccanica hamiltoniana, un sistema classico con $d$ gradi di libertà spaziali è definito su una varietà $\R^{2d}$, piú nello specifico è descritto in una varietà simplettica $2d$-dimensionale $\mathcal{M}$. Uno stato fisico è rappresentato in questa inerpretazione da un punto su $\mathcal{M}$  $x= (q^1, \dots, q^n, p_1, \dots, p_n) $. Ogni osservabile classico puó essere inerpretato come una funzione liscia $f : \mathcal{M} \to \R$, nell'esempio di prima $e_\Omega$ è associata alla funzione caratteristica dell'insieme. L'evoluzione di uno stato fisico è descritto da una curva $x(t) \in \mathcal{M}$ che soddisfa l'equazione $$\dot{x}(t)= X_H(x(t))$$ dove $X_H$ è il gradiente simplettico associato alla funzione $H \in C^\infty (\mathcal{M})$ energia meccanica. In questa generalizzazione l'equazione sopra in coordinate diventa il noto sistema di equazioni :
\begin{align*}
    \dfrac{dq^i}{dt} &= \dfrac{\partial H}{\partial p_i} \\ 
    \dfrac{dp_i}{dt} &= \dfrac{\partial H}{\partial q^i}
\end{align*}
È importante notare che l'esatta informazione iniziale di un sistema classico è un concetto di cui ha poco senso parlare a causa della finita precisione degli strumenti di misura, è possibile quindi generalizzare leggermente la teoria utilizzando al posto di un punto esatto una distribuzione di probabilità sulla varità $\mathcal{M}$ e descrivendo l'evoluazione temporale con quella che viene chiamata equazione di Liouville $$\dfrac{\partial \rho}{\partial t} + \{ \rho, H\} = 0$$ È importante notare come il meccanica classica la conoscenza di un sistema è informazionale, ossia conoscendo un esatto punto nella varità, e quindi gli esatti dati di uno stato ad un certo tempo $t$, possiamo descrivere l'evoluzione temporale (passata e futura) del sistema fisico, come vedremo dopo nel caso della meccanica quantistica questo approccio non è utilizzabile. 

Ora proveremo a ricreare una struttura simile nell'ottica della meccanica quantistica, per prima cosa è necessario dare un senso al concetto di stato in meccanica quantistica, infatti non essendo possibile "conoscere" le informazioni necessarie a descrivere un sistema quantistico completamente viene introdotto in fisica sperimentale un nuovo approccio quello operazionale. Non vengono piú misurate le proprietà di un sistema fisico viene invece preparato un sistema fisico con tali proprietà. Sarebbe come se in meccanica classica al posto di misurare posizione e momento di una particella la si preparasse in una certa posizione con un certo momento e si studiasse dopo la sua evoluzione temporale. Possiamo dare una definizione naive di stato di questo tipo "uno stato è una classe di equivalenza tra procedure di preparazione che portano il sistema in una certa condizione e che differiscono tra loro per dettagli inessenziali" è imporante notare che non è fatto ovvio cosa siano questi "dettagli inessenziali", spesso infatti vengono decisi a posteriori dell'esperimento. 

\begin{enumerate}
    \item \textbf{Spazio degli stati}\emph{: In meccanica quantistica ad ogni sistema fisico corrisponde uno spazio di Hilbert complesso separabile $\mathcal{H}$, ad ogni stato del sistema fisico corrisponde uno raggio proiettivo in $\hi$. } Negli esempi visti da noi in precedenza una particella in $\R^d$ è descritta da $\hi = \Lr$, nel caso di due particelle lo spazio di Hilbert sarà $L^2(\R^d_x \times \R^d_y)$. E abbiamo definito gli stati come $[ \psi] = \{ \lambda \psi : \lambda \in \mathbb{C}, \lambda \neq 0\} \in \mathbb{P}\hi$.
    \item \textbf{Proposizioni elementari}\emph{: Ad ogni proposizione elementare $e$ corrisponde un sottospazio lineare chiuso $\ni_e$ o equivalementemente un proiettore ortogonale autoaggiunto $E_e$ su $\hi$}
    \item \textbf{Stato dopo una misurazione} \emph{: Immediatamente dopo la misurazione di una proposizione elementare $e$ lo stato del sistema è identificato dal raggio proiettivo $\psi_e = \dfrac{\langle \psi E \psi \rangle}{\|\psi\|_2}$}
    \item \textbf{Evoluazione temporale}\emph{: Al sistema quantistico è assocaito un operatore lineare autoaggiunto $H$ su $\hi$ chiamato Hamiltoniana, l'evoluzione temporale di uno stato $\psi$ è descritta dall'equazione di Schrödinger: $$i \hbar \dfrac{d}{dt} \psi = H \psi$$}
\end{enumerate}
Vale la pena introdurre adesso un concetto che riprenderemo in futuro, abbiamo infatti dato le regole interpretative della meccanica quanstistica in un caso molto specifico quello delle proposizioni elementari su stati puri, la generalizzazione delle regole sopra prende una forma del tipo (seguendo \cite{Mor}):
\begin{enumerate}
    \item \textbf{Spazio degli stati}\emph{: Ad ogni sistema quantistico $S$ è associato un sottoinsieme del reticolo (\cite{Mor} Sez. 7.2.6) $\mathcal{L}(\hi_S)$ di proiettori ortogonali su $\hi_S$ spazio di Hilber complesso e separabile. Uno stato $\rho$ di $S$ è un operatore classe traccia positivo con traccia unitaria}
    \item \textbf{Proposizioni}\emph{: La propabilità che una proposizione $P \in \mathcal{L}(\hi_S)$ sia vera in uno stato $\rho$ è data da $\tr(\rho P)$}
    \item \textbf{Stato dopo una misurazione}\emph{: Se il sistema quantistico $S$ è in uno stato $\rho$ e una proposizione $P \in \mathcal{L}(\hi_S)$ è vera allora il sistema quantistico immediatamente dopo è: $$\rho_P = \dfrac{P \rho P}{\tr(\rho P)}$$}
    \item \textbf{Evoluazione temporale}\emph{: Al sistema quantistico $S$ è associato un operatore autoaggiunto $H$ su $\hi_S$ chiamato hamiltoniano del sistema, posto $U_t = e^{-iHt/\hbar}$ l'evoluzione temporale di uno stato $\rho_{t_0}$ è descritta dall'equazione $$\rho_{t_0 + t} = U_t \rho_{t_0} U_t^{-1}$$}
\end{enumerate}

\section*{I. Verso la meccanica ondulatoria}
Nel 1912, in Austria, si assiste a uno sviluppo significativo negli studi matematici e fisici, con particolare interesse verso la natura ondulatoria della materia e della luce.

\section*{2. Meccanica classica}
La meccanica classica pone le basi per la comprensione dei fenomeni fisici, ma presenta limiti nell'interpretare fenomeni atomici e subatomici. 

\textbf{Elettromagnetismo}: grazie agli studi di Maxwell, Hertz e Lorentz, si stabilisce che la luce è un'onda elettromagnetica. Hertz dimostra sperimentalmente che una particella carica in moto accelerato emette radiazione elettromagnetica.

\subsection*{Problemi aperti}
Nonostante i progressi, restano misteri irrisolti:
\begin{itemize}
    \item Gli spettri atomici, che mostrano righe caratteristiche per ogni elemento.
    \item La struttura atomica, la cui stabilità e organizzazione non sono spiegate dalla meccanica classica.
\end{itemize}

\section*{Mistero degli spettri atomici}
Il termine \emph{spettro}, dal latino \emph{imago} (immagine, fantasma), si riferisce alla distribuzione delle lunghezze d'onda della luce emessa o assorbita dagli atomi.

\textbf{Newton (1666)} osserva la scomposizione della luce bianca attraverso un prisma, descrivendo il fenomeno nella sua opera "Opticks" (1714).

Se al posto di luce bianca si utilizza un tubo contenente gas, come l'idrogeno, si osserva che lo spettro non è continuo ma presenta righe isolate. Questa scoperta porta Fraunhofer (1814) a inventare lo spettroscopio, uno strumento per analizzare gli spettri luminosi.

\subsection*{Ipotesi di Bunsen e Kirchhoff}
Nel 1859, Robert Bunsen e Gustav Kirchhoff formularono ipotesi fondamentali per la comprensione delle righe spettrali. Essi scoprirono che ogni elemento chimico emette luce a lunghezze d'onda specifiche quando viene riscaldato, producendo uno spettro caratteristico di righe luminose. Questo fenomeno è alla base della spettroscopia, una tecnica analitica che permette di identificare gli elementi presenti in una sostanza.

\textbf{Leggi di Kirchhoff}
Kirchhoff formulò tre leggi fondamentali della spettroscopia:
\begin{enumerate}
    \item \textbf{Prima legge}: Un corpo solido, liquido o un gas denso riscaldato emette uno spettro continuo di radiazione.
    \item \textbf{Seconda legge}: Un gas rarefatto riscaldato emette uno spettro di emissione costituito da righe luminose su uno sfondo scuro. Le posizioni delle righe sono caratteristiche dell'elemento chimico presente nel gas.
    \item \textbf{Terza legge}: Un gas freddo posto davanti a una sorgente di luce che emette uno spettro continuo produce uno spettro di assorbimento. Questo spettro è costituito da righe scure che corrispondono esattamente alle posizioni delle righe di emissione del gas.
\end{enumerate}

\textbf{Spettroscopia di Emissione e Assorbimento}
La spettroscopia di emissione e assorbimento si basa sulle leggi di Kirchhoff. Quando un elemento viene riscaldato, gli elettroni nei suoi atomi assorbono energia e saltano a livelli energetici superiori. Quando gli elettroni ritornano ai livelli energetici inferiori, emettono energia sotto forma di luce a lunghezze d'onda specifiche, producendo uno spettro di emissione.

Al contrario, quando la luce bianca passa attraverso un gas freddo, gli elettroni negli atomi del gas assorbono energia a lunghezze d'onda specifiche, saltando a livelli energetici superiori. Questo produce uno spettro di assorbimento con righe scure nelle stesse posizioni delle righe luminose dello spettro di emissione.

\textbf{Importanza delle Scoperte di Bunsen e Kirchhoff}
Le scoperte di Bunsen e Kirchhoff hanno avuto un impatto significativo sulla chimica e sull'astronomia. La spettroscopia è diventata uno strumento essenziale per l'analisi chimica, permettendo di identificare gli elementi presenti in campioni sconosciuti. In astronomia, la spettroscopia ha permesso di determinare la composizione chimica delle stelle e delle galassie, fornendo informazioni cruciali sulla loro formazione ed evoluzione.

Le ipotesi di Bunsen e Kirchhoff hanno gettato le basi per lo sviluppo della spettroscopia moderna e hanno contribuito in modo significativo alla comprensione della natura della luce e della materia.

\subsubsection*{Serie di Balmer}
Johann Balmer, nel 1885, scopre una formula empirica che descrive le lunghezze d'onda delle righe spettrali visibili dell'idrogeno. La serie di Balmer è data da:
\[
\frac{1}{\lambda} = R_H \left( \frac{1}{2^2} - \frac{1}{n^2} \right)
\]
dove \( \lambda \) è la lunghezza d'onda della luce emessa, \( R_H \) è la costante di Rydberg per l'idrogeno, e \( n \) è un numero intero maggiore di 2 (n = 3, 4, 5, ...).

Le righe della serie di Balmer si trovano nella regione visibile dello spettro elettromagnetico e sono particolarmente importanti perché furono le prime a essere osservate e studiate. Le prime quattro righe della serie di Balmer sono note come H-alfa, H-beta, H-gamma e H-delta, corrispondenti a transizioni verso il secondo livello energetico (n=2) da livelli superiori (n=3, 4, 5, 6).

\subsubsection*{Formula di Rydberg}
Johannes Rydberg generalizza la formula di Balmer per descrivere tutte le serie spettrali dell'idrogeno. La formula di Rydberg è:
\[
\frac{1}{\lambda} = R_H \left( \frac{1}{n_1^2} - \frac{1}{n_2^2} \right)
\]
dove \( n_1 \) e \( n_2 \) sono numeri interi tali che \( n_2 > n_1 \). Questa formula permette di calcolare le lunghezze d'onda delle transizioni elettroniche tra diversi livelli energetici dell'atomo di idrogeno.

La formula di Rydberg può essere utilizzata per descrivere altre serie spettrali dell'idrogeno, come la serie di Lyman (ultravioletto), la serie di Paschen (infrarosso), la serie di Brackett e la serie di Pfund. Ogni serie è caratterizzata da un valore specifico di \( n_1 \):
\begin{itemize}
    \item Serie di Lyman: \( n_1 = 1 \)
    \item Serie di Balmer: \( n_1 = 2 \)
    \item Serie di Paschen: \( n_1 = 3 \)
    \item Serie di Brackett: \( n_1 = 4 \)
    \item Serie di Pfund: \( n_1 = 5 \)
\end{itemize}

La scoperta delle serie spettrali e la formulazione della legge di Rydberg furono fondamentali per lo sviluppo della teoria quantistica e per la comprensione della struttura atomica.
La serie di Balmer si verifica quando un elettrone in un atomo di idrogeno cade da un livello energetico superiore a \( n = 2 \). Questo avviene perché gli elettroni negli atomi possono occupare solo livelli energetici discreti. Quando un elettrone assorbe energia, può saltare a un livello energetico superiore. Successivamente, quando l'elettrone ritorna a un livello energetico inferiore, emette energia sotto forma di luce. 

Nel caso della serie di Balmer, l'elettrone cade verso il secondo livello energetico (\( n = 2 \)) da livelli superiori (\( n > 2 \)). Le transizioni che coinvolgono il secondo livello energetico producono righe spettrali nella regione visibile dello spettro elettromagnetico. Questo è il motivo per cui la serie di Balmer è particolarmente importante e fu una delle prime ad essere osservata e studiata.

Le righe della serie di Balmer sono visibili perché la differenza di energia tra i livelli coinvolti corrisponde a lunghezze d'onda che rientrano nella gamma della luce visibile. Questo rende la serie di Balmer fondamentale per la spettroscopia e per la comprensione della struttura atomica.

\section*{Mistero della struttura atomica}
\textbf{Ludwig Boltzmann e Ernst Mach}: sostengono l'ipotesi atomica e l'energetismo. La materia può essere descritta come composta da unità discrete (atomi o molecole).

La stechiometria introduce il concetto di mole come unità della materia. L'equazione di Boltzmann diventa la base per le teorie cinetiche dei gas.

\section*{Modelli atomici del 1900}

Nel corso del 1900, la comprensione della struttura atomica ha subito una serie di evoluzioni significative grazie ai contributi di diversi scienziati. Tra i modelli più influenti vi sono quelli proposti da J.J. Thomson, Ernest Rutherford e Niels Bohr.

\subsection*{Modello di Thomson}
Nel 1897, J.J. Thomson scoprì l'elettrone, una particella subatomica carica negativamente. Basandosi su questa scoperta, Thomson propose il cosiddetto "modello a panettone" o "modello a cocomero" dell'atomo. In questo modello, gli elettroni erano dispersi all'interno di una sfera carica positivamente, simile ai semi di un cocomero nella polpa. Questo modello spiegava la neutralità elettrica dell'atomo, ma non riusciva a spiegare i dettagli degli spettri atomici.

\subsection*{Modello di Rutherford}
Nel 1911, Ernest Rutherford, attraverso il famoso esperimento della lamina d'oro, propose un nuovo modello atomico. Rutherford scoprì che la maggior parte delle particelle alfa passava attraverso la lamina d'oro senza deviazioni significative, ma alcune venivano deviate con angoli elevati e alcune addirittura rimbalzavano indietro. Da queste osservazioni, Rutherford dedusse che la carica positiva dell'atomo e la maggior parte della sua massa erano concentrate in un piccolo nucleo centrale, mentre gli elettroni orbitavano attorno a questo nucleo. Questo modello, tuttavia, non spiegava la stabilità degli atomi né gli spettri di emissione.
Il modello di Rutherford implica che gli elettroni dovrebbero spiraleggiare verso il nucleo in tempi brevissimi (\(\sim 10^{-10}\) s) per via della legge di Hertz. Questo contrasta con la stabilità osservata della materia.


\subsection*{Modello di Bohr}
Nel 1913, Niels Bohr propose un modello atomico che combinava le idee di Rutherford con i concetti della teoria quantistica. Bohr suggerì che gli elettroni orbitano attorno al nucleo in orbite stazionarie senza emettere energia. Gli elettroni possono saltare da un'orbita all'altra emettendo o assorbendo quanti di energia, con la frequenza della radiazione emessa data dalla differenza di energia tra le orbite divisa per la costante di Planck. Questo modello spiegava con successo gli spettri di emissione dell'idrogeno e introduceva il concetto di quantizzazione dell'energia negli atomi.
Il modello di Bohr rappresenta un passo cruciale nella comprensione della struttura atomica, introducendo concetti che sarebbero stati fondamentali per la meccanica quantistica. Ecco alcuni dettagli chiave del modello di Bohr:

\subsubsection*{Postulati di Bohr}
Bohr formulò tre postulati principali per descrivere il comportamento degli elettroni negli atomi:
\begin{enumerate}
    \item \textbf{Orbite stazionarie}: Gli elettroni orbitano attorno al nucleo in orbite stazionarie senza emettere radiazione elettromagnetica. Queste orbite sono quantizzate, cioè solo certe orbite con specifiche energie sono permesse.
    \item \textbf{Quantizzazione del momento angolare}: Il momento angolare dell'elettrone in un'orbita stazionaria è quantizzato ed è dato da \( L = n\hbar \), dove \( n \) è un numero intero positivo (numero quantico principale) e \( \hbar \) è la costante di Planck ridotta.
    \item \textbf{Transizioni quantiche}: Gli elettroni possono saltare da un'orbita stazionaria a un'altra solo assorbendo o emettendo un quanto di energia \( E = h\nu \), dove \( h \) è la costante di Planck e \( \nu \) è la frequenza della radiazione emessa o assorbita.
\end{enumerate}

\subsubsection*{Energia degli Elettroni}
L'energia totale di un elettrone in un'orbita stazionaria è data dalla somma dell'energia cinetica e dell'energia potenziale. Per l'atomo di idrogeno, l'energia dell'elettrone in un'orbita con numero quantico principale \( n \) è:
\[
E_n = - \frac{13.6 \text{ eV}}{n^2}
\]
dove \( 13.6 \text{ eV} \) è l'energia di ionizzazione dell'idrogeno. Questa formula mostra che l'energia degli elettroni negli atomi è quantizzata e diminuisce all'aumentare del numero quantico principale \( n \).
Il modello di Bohr spiega con successo gli spettri di emissione e assorbimento dell'idrogeno predetti dalla formula di Rydberg. 

\subsubsection*{Limitazioni del Modello di Bohr}
Nonostante il successo nel descrivere l'atomo di idrogeno, il modello di Bohr presenta alcune limitazioni:
\begin{itemize}
    \item Non riesce a spiegare gli spettri di atomi più complessi dell'idrogeno.
    \item Non tiene conto delle interazioni tra elettroni in atomi con più di un elettrone.
    \item Non spiega la struttura fine degli spettri atomici, che richiede una descrizione relativistica degli elettroni.
\end{itemize}
Il modello di Bohr ha avuto un impatto duraturo sulla fisica, ponendo le basi per lo sviluppo della meccanica quantistica. I concetti di quantizzazione dell'energia e delle orbite stazionarie sono stati fondamentali per la successiva teoria quantistica degli atomi, sviluppata da Schrödinger, Heisenberg e altri.

Svilupperemo ora i calcoli di Bohr per l'atomo di idrogeno, che portano alla formula di Rydberg e alla spiegazione della serie di Balmer. Partiamo dalla quantizzazione dei raggi delle orbite stazionarie nell'idrogeno. La legge di Newton in sola presenza di forza elettrostatica diventa: (ponendo le costanti dielettriche ad 1) 
$$m\frac{v^2}{r} = \frac{e^2}{r^2}$$e poichè $$L = mvr = n\hbar$$ otteniamo $$v = \frac{e^2}{n\hbar}$$ e sostituendo in $$m\frac{v^2}{r} = \frac{e^2}{r^2}$$ otteniamo $$r = \frac{n^2\hbar^2}{me^2}$$ che è la quantizzazione dei raggi delle orbite stazionarie. Sostituendo in ricordando che $$E_{tot}= E_{pot} + E_{cin} = -\frac{e^2}{2r} +  m \dfrac{v^2}{2} = -\frac{e^2}{r} + \frac{e^2}{2r} = -\frac{e^2}{2r}$$ otteniamo $$E_n = -\frac{e^2}{2r} = -\frac{e^2}{2} \frac{me^2}{n^2\hbar^2} = -\frac{me^4}{2n^2\hbar^2}$$ che è l'energia delle orbite stazionarie. Infine, la differenza di energia tra due orbite stazionarie è data da $$\Delta E = E_{n_2} - E_{n_1} = \frac{me^4}{2\hbar^2} \left( \frac{1}{n_1^2} - \frac{1}{n_2^2} \right)$$ che è la formula di Rydberg, infine ricordando semplicemente l'equazione di De Broglie $E = h \ni$ otteniamo le varie frequenze emesse nei salti energetici. 


D'ora in poi nel resto delle note seguiremo la seguente notazione:
\begin{itemize}
    \item \textbf{Spazio delle posizioni}: $\mathbb{R}^d \simeq E^d \ni x,\, y,\,z$ con $[x] = [y] = [z] = L \text{ lunghezza}$.
    \item \textbf{Spazio dei vettori d'onda}: $\mom\simeq \hat{E}^d \ni k, \, \xi, \, \alpha , \, \beta$ con $[k] = [ \xi ] = [\alpha] = [\beta] = L^{-1} \text{ inverso della lunghezza}$.
\end{itemize}

\begin{definition}
    Sia $f \in L^1(\mathbb{R}^d)$, chiameremo trasformata di Fourier e antitrasformata di Fourier le mappe lineari da $L^1(\mathbb{R}^d) $ a $ L^\infty(\mathbb{R}^d)$ definite rispettivamente da: 
    \begin{align*}
      (\mathcal{F} f)(k) &= \hat{f}(k) = \dfrac{1}{(2 \pi)^\frac{d}{2}}\int_{\mathbb{R}^d} e^{-i k \cdot x} f(x)  dx \\
    (\mathcal{F}_- f)(x) &= \dfrac{1}{(2 \pi)^\frac{d}{2}}\int_{\mathbb{R}^d} e^{i k \cdot x} f(k)  dk \\
    \end{align*}

\end{definition}

Definiremo le seguenti applicazioni, dati $a \in \mathbb{R}^d$,  $\alpha \in \mom$ e $\lambda \in \mathbb{R}$:
\begin{itemize}
    \item \textbf{Traslazioni}: $(\tau_a f)(x) = f(x - a)$.
    \item \textbf{Moltiplicazione per carattere}: $(c_\alpha f)(x) = e^{i \alpha \cdot k} f(x)$.
    \item \textbf{Dilatazioni/Contrazioni}: $(D_\lambda f)(x) = \lambda^{-d/2} f(\dfrac{x}{\lambda})$.
\end{itemize}

\begin{proposition}
 La formula sopra definisce una mappa lineare continua $F: L^1(\mathbb{R}^d) \to C_b(\mathbb{R}^d) \cap L^\infty(\mathbb{R}^d)$ tale che:
\begin{itemize}
    \item $\mathcal{F}(\tau_a f)(k) = e^{-i k \cdot a} \hat{f}(k)$.
    \item $\mathcal{F}(c_\alpha f)(k) = \hat{f}(k - x)$.
    \item $\mathcal{F}(D_\lambda f)(k) = \lambda^{d/2} \hat{f}(a k)= D_\frac{1}{\lambda} \hat{f}(k)$.
\end{itemize}
\end{proposition}

\begin{proof} *
    La dimostrazione è abbastanza semplice, per dimostrare che è in $C_b$ basta vedere che $\| \hat{f}\|_\infty \leq \| f\|_1$ e per dimostrare che è continua basta vedere che data $\xi_j \to \xi$ convergente allora $e^{-i \xi_j \cdot x} f(x) \to e^{-i \xi \cdot x} f(x)$ quasi ovunque e inotre $|e^{-i \xi_j \cdot x} f(x)| \leq |f(x)| \in L^1(\mathbb{R}^d)$, il resto segue facendo i conti.
\end{proof}


\begin{proposition}[Lemma di Rienmann-Lebesgue]
    Sia $f \in L^1(\mathbb{R}^d)$, allora $\hat{f}(k)$ tende a $0$ per $|k| \to \infty$.
\end{proposition}



\section*{2. Estensione della teoria su $L^2(\mathbb{R}^d)$}
Il problema è che nella teoria in $L^1$ l'inversa della trasformata di Fourier non esiste, per ovviare a questo problema prima si studia la restrizione della trasformata allo spazio di Schwarts $\Sw$ dove trasformata e antitrasformata di Fourier sono una l'inversa dell'altra si passa alle classi di equivalenza e infine si estende per linearità e continuità a $L^2(\mathbb{R}^d)$, il che fa si, poichè $\mathcal{F}\mathcal{F}_- = \un_{\Sw}$ e che l'estensione dell'indentià è unica vale anche $\mathcal{F}\mathcal{F}_- = \un_{L^2}$. In maniera formale avremo che:
\begin{theorem}
    Lo spazio di Schwartz è invariante sotto l'azione della trasfromata e l'antitrasformata di Fourier, le due trasformazioni ristrette a $\Sw$ sono una l'inversa dell'altra e sono isometrie per il prodotto scalare di $L^2(\mathbb{R}^d)$. \label{thm:Fourier}
\end{theorem}

La seguente dimostrazione è presa da \cite{Mor}

\begin{proof} *     
Dimostriamo ora per $\mathcal{F}$, il risultato per $\mathcal{F}_{-}$ è analogo.

È immediato verificare che:
\begin{equation*}
    |\partial_k^\alpha e^{i k \cdot x} f(x)| = |i^{|\alpha|} M_\alpha(x) f(x)| \leq |M_\alpha(x) f(x)|,
\end{equation*}
con $M_\alpha(x) = x^\alpha f(x)$ l'operatore moltiplicativo. 

Poiché $f \in S(\mathbb{R}^n)$, segue che $M_\alpha(x) f(x) \in L^1(\mathbb{R}^n)$. Usando il teorema della convergenza dominata di Lebesgue, possiamo scambiare derivata e integrale:
\begin{equation*}
    \partial_k^\alpha \hat{f}(k) = i^{|\alpha|} \int_{\mathbb{R}^n} e^{i k \cdot x} \frac{1}{(2\pi)^{n/2}} M_\alpha(x) f(x) dx.
\end{equation*}


Notando che $f$ si annulla più rapidamente di ogni potenza inversa di $|x|$ per $|x| \to +\infty$, si trova che:
\begin{equation*}
M_{{\beta}}({k}) \hat{f}(k) = \int_{\mathbb{R}^n} (-i)^{|{\beta}|} \partial^{{\beta}}_{{x}} \left( \frac{e^{i{k} \cdot {x}}}{(2\pi)^{n/2}} \right) f({x}) \, \dd {x},
\end{equation*}
e quindi, usando l'integrazione per parti:
\begin{equation*}
M_{{\beta}}({k}) \hat{f}(k) = i^{|{\beta}|} \int_{\mathbb{R}^n} \frac{e^{i{k} \cdot {x}}}{(2\pi)^{n/2}} \partial^{{\beta}}_{{x}} f({x}) \, \dd {x}.
\end{equation*}

Quindi, inserendo $\partial^{{\alpha}}_{{k}} g$ al posto della funzione $g$ in (3.64) e tenendo conto di (a), si ha:
\begin{equation*}
|M_{{\beta}}({k}) \partial^{{\alpha}}_{{k}} \hat{f}(k)| \leq \left| \partial^{{\beta}} (M_{{\alpha}} f) \right|_1,
\end{equation*}
per ogni ${k} \in \mathbb{R}^n$. Essendo finito il secondo membro, in quanto $f \in \mathcal{S}(\mathbb{R}^n)$, ed essendo ${\alpha}$ e ${\beta}$ arbitrari, concludiamo che $\hat{f} \in \mathcal{S}(\mathbb{R}^n)$.

Possiamo riscrivere le equazioni sopra come:
\begin{align*}
\partial^{{\alpha}} \mathcal{F} &= i^{|{\alpha}|} \mathcal{F} M_{{\alpha}},  \\
M_{{\beta}} \mathcal{F} &= i^{|{\beta}|} \mathcal{F} \partial^{{\beta}}, 
\end{align*}
dove $\mathcal{F}$ è in realtà la restrizione della trasformata di Fourier a $\mathcal{S}(\mathbb{R}^n)$. Osservando che:
\begin{equation*}
\mathcal{F} h = \mathcal{F}_{-} h \quad \text{per ogni } h \in \mathcal{S}(\mathbb{R}^n),
\end{equation*}
si ricava facilmente:
\begin{align*}
\partial^{{\alpha}} \mathcal{F}_{-} &= (-1)^{|{\alpha}|} i^{|{\alpha}|} \mathcal{F}_{-} M_{{\alpha}},  \\
M_{{\beta}} \mathcal{F}_{-} &= (-1)^{|{\beta}|} i^{|{\beta}|} \mathcal{F}_{-} \partial^{{\beta}}. 
\end{align*}

Abbiamo dimostrato in particolare che valgono le seguenti:
\begin{align*}
\mathcal{F}\mathcal{F}_{-} M_{{\alpha}} &= M_{{\alpha}} \mathcal{F}\mathcal{F}_{-},  \\
\mathcal{F}_{-}\mathcal{F} M_{{\alpha}} &= M_{{\alpha}} \mathcal{F}_{-}\mathcal{F}, 
\end{align*}
 e anche
\begin{align*}
\mathcal{F}\mathcal{F}_{-} \partial^{{\alpha}} &= \partial^{{\alpha}} \mathcal{F}\mathcal{F}_{-},  \\
\mathcal{F}_{-}\mathcal{F} \partial^{{\alpha}} &= \partial^{{\alpha}} \mathcal{F}_{-}\mathcal{F}. 
\end{align*}

Mostreremo ora che, in virtù di tali relazioni di commutazione, gli operatori
\begin{equation*}
J := \mathcal{F}\mathcal{F}_{-} \quad \text{e} \quad J_{-} := \mathcal{F}_{-}\mathcal{F}
\end{equation*}
devono essere l'operatore identità di $\mathcal{S}(\mathbb{R}^n)$. Per prima cosa proviamo che, fissati ${x}_0 \in \mathbb{R}^n$ e $f \in \mathcal{S}(\mathbb{R}^n)$, il valore di $(Jf)({x}_0)$ dipende solo da $f({x}_0)$. Se $f \in \mathcal{S}(\mathbb{R}^n)$ possiamo sempre scrivere:
\begin{equation*}
f({x}) = f({x}_0) + \int_0^1 \dv{f}{t}({x}_0 + t({x} - {x}_0)) \dd{t} = f({x}_0) + \sum_{i=1}^n (x_i - x_{0i}) g_i({x}),
\end{equation*}
dove le funzioni $g_i$ (che sono $C^\infty(\mathbb{R}^n)$, come si verifica facilmente) sono definite da:
\begin{equation*}
g_i({x}) := \pdv{}{x_i} \int_0^1 f({x}_0 + t({x} - {x}_0)) \dd{t}.
\end{equation*}

Applicando $J$ ad ambo i membri di questa decomposizione e tenendo conto del fatto che $J$ commuta con i polinomi in ${x}$ per (3.69), otteniamo:
\begin{equation*}
(Jf_1)({x}) = (Jf_2)({x}) + \sum_{i=1}^n (x_i - x_{0i}) (Jg_i)({x}).
\end{equation*}
Prendendo ${x} = {x}_0$, si vede che $(Jf_1)({x}_0) = (Jf_2)({x}_0)$ sotto l'ipotesi iniziale $f_1({x}_0) = f_2({x}_0)$. Quindi, come detto, $(Jf)({x}_0)$ è una funzione soltanto di $f({x}_0)$. Tale funzione deve essere anche lineare, dato che $J$ è lineare per costruzione. Ne consegue che sarà:
\begin{equation*}
(Jf)({x}_0) = j({x}_0) f({x}_0),
\end{equation*}
dove $j$ è una funzione su $\mathbb{R}^n$ a valori in $\mathbb{C}$. Poichè $x_0$ è arbitrario allora abbiamo provato che $J$ agisce come la moltiplicazione per una funzione $j$. È poi immediato verificare che tutte le derivate di $j$ sono nulle e calcolando la trasformata di Fourier della gaussiana si ottiene che tale costante è proprio $1$.

\end{proof}

\begin{theorem}[Estensione limitata]
    Siano $E_1, E_2$ spazi di Banach, $D$ un sottoinsieme denso di $E_1$ e $T: D \to E_2$ un operatore lineare prelimitato(ossia $\|T\psi\| \leq c \|\psi\|$ per ogni $\psi \in D$). Allora esiste un'unica estensione lineare e continua $\tilde{T}: E_1 \to E_2$ tale che $\|\tilde{T}\psi\| \leq c \|\psi\|$ 
\end{theorem}

La dimostrazione di questo teorema è nota. Ma ci permette ora, sapendo che $\|\mathcal{F}_{\Sw} \psi \|_{L^2} = \|\psi\|_{L^2}$ per ogni $\psi \in \Sw$ di estendere la trasformata di Fourier a tutto $L^2(\mathbb{R}^d)$ e di estendere la sua inversa, l'antitrasformata, su $\Sw$ all'inversa su $L^2(\mathbb{R}^d)$.

\begin{theorem}[Fourier-Plancherel]
    La trasformata di Fourier si estende per linearità e continutià ad una mappa lineare $\mathcal{F} : L^2(\mathbb{R}^d) \to L^2(\mathbb{R}^d)$ tale che: 
\begin{enumerate}
    \item Preserva il prodotto hermitiano.
    \item È una isometria.
    \item È suriettiva.
\end{enumerate}
\end{theorem}
\begin{proof}* 
    La dimostrazione è immediata dal Teorema \ref{thm:Fourier} e dai commenti ad inizio sezione.
\end{proof}

\section*{3. Ottica ondulatoria ed equazione delle onde}

\begin{definition}
Chiameremo equazione delle onde il sistema: 
    \begin{align*}
        \Box u = \frac{\partial^2 u}{\partial t^2} - \Delta u = 0,
    \end{align*}
    con condizioni iniziali $u(0, x) = u_0(x)$, $\frac{\partial u}{\partial t}(0, x) = v_0(x)$.
Incognita: $u: \mathbb{R}\times \mathbb{R}^d \to \mathbb{C}$ tale che $(t,x) \mapsto u(t, x)$.
\end{definition}

Una \emph{soluzione} dell'equazione delle onde è una funzione $u: \mathbb{R} \to L^2(\mathbb{R}^d) = \hi$ che a $t$ associa $u(t, \cdot)$. Scegliamo $u(t, \cdot)$ a valori in $L^2(\mathbb{R}^d)$ perchè è uno spazio di Hilbert ed è dotato di na buona teoria di Fouier. Una buona soluzione deve essere tale che: 
\begin{itemize}
    \item Deve avere regolarità nel tempo: $$ \partial_t u(t, \cdot) = L^2 -\lim_{h \to 0} \frac{u(t+h, \cdot) - u(t, \cdot)}{h}$$ se ipotizziamo una funzione $g$ come derivata prima dobbiamo controllare che $$ \| \frac{u(t+h , \cdot )- u (t,\cdot)}{h} - g \|_{L^2} \to 0 $$, si parla quindi di derivabilità quasi ovunque.
    \item $\dfrac{1}{c^2}\dfrac{\partial^2 u}{\partial t^2} = \Delta u$ notiamo subito che non è sempre vero che $\Delta u \in L^2$ introduciamo quindi lo spazio $\hi^s(\mathbb{R}^d):= \{ f \in \Lr : \int_{\mom}(1 + |k|^2)^s |\hat{f}(k)|^2 dk < \infty \}$ con $s \in \mathbb{R}$ è possibile dimostrare che la $s$-esima derivata distribuzionale è integrabile se e solo se $f \in \hi^s(\mathbb{R}^d)$. Questi spazi sono inscatolati tra loro e sono tutti densi in $L^2(\mathbb{R}^d)$, vale quindi $\Lr \supseteq \hi^0(\mathbb{R}^d) \supseteq \hi^1(\mathbb{R}^d) \supseteq \hi^2(\mathbb{R}^d) \supseteq \ldots$.
    
\end{itemize}





\include{Chap2}
\chapter{Monaco}


Chiameremo $\Bo$ l'insieme degli operatori compatti su $\hi$.

\begin{definition}
    Sia $\hi$ con la norma indotta $\| \|$ dal prodotto scalare, diremo che $A \in \B$ è un operatore di Hilber-Schmidt se esiste una base hilbertiana $\{ u_k\}$ tale che $\sum \|Au_k \|<\infty$. 
\end{definition}

Indicheremo la classe di operatori di Hilber-Schmidt su $\hi$ come $\Hs$, è possibile inoltre dimostrare che 





\begin{definition}
    Una mappa lineare $\Phi : \B \to \mathbb{C}$ è detta \emph{ una famiglia di valori attesi} se valgono le seguenti:

\begin{itemize}
    \item $\Phi(\un) = 1$.
    \item $\Phi(A)$ è reale quando $A$ è autoaggiunto.
    \item $\Phi(A)$ è positivo quando $A$ è autoaggiunto e positivo.
    \item Per ogni successione $A_n \in \B$ se $\| A_n\psi - A \psi\| \to 0$ per tutti $\psi \in \hi$ allora $\Phi(A_n) \to \Phi(A)$.
\end{itemize}
\end{definition}

\begin{definition}
    Un operatore $\rho\in \Ct$ è una matrice densità se $\rho$ è autoaggiunto, non negativo e vale $\Tr \rho =1$.
\end{definition}

\begin{definition}
    Siano $\hi_1, \hi_2$ spazi di Hilber definiremo il \emph{prodotto tensore tra due spazi di Hilbert} $\hi_1 \hat{\otimes} \hi_2$ come il completamento di $\hi_1 \otimes \hi_2$ rispetto al prodotto:
$(u_1 \otimes v_1, u_2 \otimes v_2)= (u_1, u_2)_1 (v_1, v_2)_2$.
\end{definition}



\begin{proposition}
    L'applicazione $ L^2(X_1, \mu_1) \Ht L^2(X_2, \mu_2) \to L^2(X_1 \times X_2, \mu_1 \times \mu_2) $ è un isomorfismo.
\end{proposition}

Per la dimostrazione di veda \cite{Hall}.

Se ho n sistemi quantistici composti descritti da $\hi_1, \dots, \hi_n$ allora il sistema composto è descritto dallo spazio di hilber $\hi = \hi_1 \Ht \dots \Ht \hi_n$. 
Per cui un sistema quantistico di n particelle puó essere descritto da $L^2(\mathbb{R}^{nd})= \Lr \Ht \dots \Ht \Lr$. Ora vogliamo generalizzare gli operatori a questi spazi tensore, 
siano $A_i^* = A_i\in \mathfrak{B}(\hi_1)$ definiremo l'osservabile del sistema composto come $A(\psi_1 \otimes \dots \otimes \psi_n)= A_1 \psi_1 \otimes \dots \otimes A_n \psi_n$.
In particolare:
\begin{enumerate}
    \item Se $A_i^* = A_i\in \mathfrak{B}(\hi_i)$ definisco $\B \ni A^{ (i)} := \un \otimes \un \otimes \dots \otimes \un \otimes A_i \otimes \un \otimes \dots \otimes \un$.
    \item Se $I = \{ i_1, \dots , i_k \} \subset \{1, \dots , n \}$ e $A \in \mathfrak{B}(\hi_{i_1}\Ht \dots \Ht \hi_{i_k})$ allora $A^{ (I)} \in \B$ agisce sui sottosistemi associati.
\end{enumerate}

\begin{theorem}
    Se $\rho^* = \rho \geq 0$ $\Tr(\rho) = 1$ matrice densità su $\hi$ allora siste una matrice densità $\rho^(I)$ su $\mathfrak{B}(\hi_{i_1}\Ht \dots \Ht \hi_{i_k})$ tale che 
\end{theorem}







\begin{thebibliography}{20}

\bibitem{Mor}
V. Moretti (2012) \emph{Teoria spettrale e meccanica quantistica}, Springer.

\bibitem{Hall}
Brian C. Hall (2013) \emph{Quantum Theory for Mathematicians}, Springer New York. 

\bibitem{Bir}
Birman, M.S., Solomjak, M.Z. (1987) \emph{Spectral Theory of Self-Adjoint Operators in Hilbert Space}, D. Reidel Publishing Company, Dordrecht.

\bibitem{Humphreys} 
James E. Humphreys (1972) \emph{Introduction to Lie Algebras and Representation Theory}, Springer.


\end{thebibliography}




\end{document}


