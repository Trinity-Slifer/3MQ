\documentclass[12pt,a4paper]{report}
\usepackage{graphicx}
\usepackage[margin=1in]{geometry}
\usepackage{faktor}
\usepackage{amsmath}
\usepackage[makeroom]{cancel}
\usepackage{mathtools}
\usepackage{physics}
\usepackage{siunitx}
\usepackage{amsthm}
\usepackage{dynkin-diagrams}
\usepackage{wrapfig}
\usepackage{subcaption}
\usepackage{amssymb}
\usepackage{dsfont}
\usepackage{float}
\usepackage{booktabs}
\usepackage{multirow}
\usepackage[T1]{fontenc}
\usepackage[utf8]{inputenc}
\usepackage[italian]{babel}
\usepackage{newlfont}
\usepackage{color}

\theoremstyle{definition}
\newtheorem{definition}{Definizione}[section]
\newtheorem{example}[definition]{Esempio}
\newtheorem{observation}[definition]{Osservazione}
\numberwithin{equation}{section}
\theoremstyle{plain}
\newtheorem{theorem}[definition]{Teorema}
\newtheorem{proposition}[definition]{Proposizione}
\newtheorem{corollary}[definition]{Corollario}
\newtheorem{lemma}[definition]{Lemma}
\theoremstyle{remark}
\newtheorem{remark}[definition]{Remark}
\newtheorem{postulato}[definition]{Postulato}
\input xy
\xyoption{all}


\DeclareMathOperator\Sym{Sym}
\DeclareMathOperator\End{End}
\DeclareMathOperator\Aut{Aut}
\DeclareMathOperator\Der{Der}
\DeclareMathOperator\ad{ad}
\DeclareMathOperator\Ker{Ker}
\DeclareMathOperator\Ran{Ran}
\DeclareMathOperator\Hom{Hom}
\DeclareMathOperator\gr{gr}
\DeclareMathOperator\mult{mult}
\DeclareMathOperator\Sw{ \mathcal{S}(\mathbb{R}^n) }
\DeclareMathOperator\mom{ \hat{\mathbb{R}^n} }
\DeclareMathOperator\Cc{ C^\infty_c(\mathbb{R}^n) }
\DeclareMathOperator\Lr{ L^2(\mathbb{R}^n) }
\DeclareMathOperator\B{ \mathfrak{B}(\mathcal{H}) }
\DeclareMathOperator\Ct{ \mathfrak{B}_1(\mathcal{H}) }
\DeclareMathOperator\Hs{ \mathfrak{B}_2(\mathcal{H}) }
\DeclareMathOperator\Bo{ \mathfrak{B}_\infty(\mathcal{H}) }
\DeclareMathOperator\dt{ \mathcal{D}(T)}
\DeclareMathOperator\ex{ \mathcal{E}}
\DeclareMathOperator\hi{\mathcal{H}}
\DeclareMathOperator\Ht{\hat{\otimes}}
\DeclareMathOperator\un{\mathds{1}}



\begin{document}


\title{Dispense 3MQ}
\date{Gennaio 2025}
\maketitle
\vspace{8.0cm}

Le dimostrazioni segnate con * non sono state svolte a lezione e alle volte richiederanno teoremi e metodi non visti, a meno che non sia specificato diversamente è possibile trovare tutto in \cite{Mor}.
$\B \; \Cc \Sw$
\vspace{1.0cm}


\newpage
\tableofcontents 
\newpage


D'ora in poi nel resto delle note seguiremo la seguente notazione:
\begin{itemize}
    \item \textbf{Spazio delle posizioni}: $\mathbb{R}^d \simeq E^d \ni x,\, y,\,z$ con $[x] = [y] = [z] = L \text{ lunghezza}$.
    \item \textbf{Spazio dei vettori d'onda}: $\mom\simeq \hat{E}^d \ni k, \, \xi, \, \alpha , \, \beta$ con $[k] = [ \xi ] = [\alpha] = [\beta] = L^{-1} \text{ inverso della lunghezza}$.
\end{itemize}

\begin{definition}
    Sia $f \in L^1(\mathbb{R}^d)$, chiameremo trasformata di Fourier e antitrasformata di Fourier le mappe lineari da $L^1(\mathbb{R}^d) $ a $ L^\infty(\mathbb{R}^d)$ definite rispettivamente da: 
    \begin{align*}
      (\mathcal{F} f)(k) &= \hat{f}(k) = \dfrac{1}{(2 \pi)^\frac{d}{2}}\int_{\mathbb{R}^d} e^{-i k \cdot x} f(x)  dx \\
    (\mathcal{F}_- f)(x) &= \dfrac{1}{(2 \pi)^\frac{d}{2}}\int_{\mathbb{R}^d} e^{i k \cdot x} f(k)  dk \\
    \end{align*}

\end{definition}

Definiremo le seguenti applicazioni, dati $a \in \mathbb{R}^d$,  $\alpha \in \mom$ e $\lambda \in \mathbb{R}$:
\begin{itemize}
    \item \textbf{Traslazioni}: $(\tau_a f)(x) = f(x - a)$.
    \item \textbf{Moltiplicazione per carattere}: $(c_\alpha f)(x) = e^{i \alpha \cdot k} f(x)$.
    \item \textbf{Dilatazioni/Contrazioni}: $(D_\lambda f)(x) = \lambda^{-d/2} f(\dfrac{x}{\lambda})$.
\end{itemize}

\begin{proposition}
 La formula sopra definisce una mappa lineare continua $F: L^1(\mathbb{R}^d) \to C_b(\mathbb{R}^d) \cap L^\infty(\mathbb{R}^d)$ tale che:
\begin{itemize}
    \item $\mathcal{F}(\tau_a f)(k) = e^{-i k \cdot a} \hat{f}(k)$.
    \item $\mathcal{F}(c_\alpha f)(k) = \hat{f}(k - x)$.
    \item $\mathcal{F}(D_\lambda f)(k) = \lambda^{d/2} \hat{f}(a k)= D_\frac{1}{\lambda} \hat{f}(k)$.
\end{itemize}
\end{proposition}

\begin{proof} *
    La dimostrazione è abbastanza semplice, per dimostrare che sta in $C_b$ basta vedere che $\| \hat{f}\|_\infty \leq \| f\|_1$ e per dimostrare che è continua basta vedere che data $\xi_j \to \xi$ convergente allora $e^{-i \xi_j \cdot x} f(x) \to e^{-i \xi \cdot x} f(x)$ quasi ovunque e inotre $|e^{-i \xi_j \cdot x} f(x)| \leq |f(x)| \in L^1(\mathbb{R}^d)$, il resto segue facendo i conti.
\end{proof}


\begin{proposition}[Lemma di Rienmann-Lebesgue]
    Sia $f \in L^1(\mathbb{R}^d)$, allora $\hat{f}(k)$ tende a $0$ per $|k| \to \infty$.
\end{proposition}



\section*{2. Estensione della teoria su $L^2(\mathbb{R}^d)$}
Il problema è che nella teoria in $L^1$ l'inversa della trasformata di Fourier non esiste, per ovviare a questo problema prima si studia la restrizione della trasformata allo spazio di Schwarts $\Sw$ dove trasformata e antitrasformata di Fourier sono una l'inversa dell'altra si passa alle classi di equivalenza e infine si estende per linearità e continuità a $L^2(\mathbb{R}^d)$, il che fa si, poichè $\mathcal{F}\mathcal{F}_- = \un_{\Sw}$ e che l'estensione dell'indentià è unica vale anche $\mathcal{F}\mathcal{F}_- = \un_{L^2}$. In maniera formale avremo che:
\begin{theorem}
    
\end{theorem}

 Dimostriamo ora per $\mathcal{F}$, il risultato per $\mathcal{F}^{-1}$ è analogo.

Sia
\begin{equation*}
    g(k) := \int_{\mathbb{R}^n} e^{i k \cdot x} \frac{1}{(2\pi)^{n/2}} f(x) dx.
\end{equation*}

\textbf{Passaggio della derivata sotto il segno dell'integrale:} È facile verificare che:
\begin{equation*}
    |\partial_k^\alpha e^{i k \cdot x} f(x)| = |i^{|\alpha|} M_\alpha(x) f(x)| \leq |M_\alpha(x) f(x)|,
\end{equation*}
con $M_\alpha(x) = x^\alpha f(x)$. 

Poiché $f \in S(\mathbb{R}^n)$, segue che $M_\alpha(x) f(x) \in L^1(\mathbb{R}^n)$. Usando il teorema della convergenza dominata di Lebesgue, possiamo scambiare derivata e integrale:
\begin{equation*}
    \partial_k^\alpha g(k) = i^{|\alpha|} \int_{\mathbb{R}^n} e^{i k \cdot x} \frac{1}{(2\pi)^{n/2}} M_\alpha(x) f(x) dx.
\end{equation*}

\section*{3. Ottica ondulatoria ed equazione delle onde}
\begin{itemize}
    \item Incognita: $u(t, x)$ con $(t, x) \in \mathbb{R} \times \mathbb{R}^d$.
    \item Equazione delle onde:
    \begin{align*}
        \Box u = \frac{\partial^2 u}{\partial t^2} - \Delta u = 0,
    \end{align*}
    con condizioni iniziali $u(0, x) = u_0(x)$, $\frac{\partial u}{\partial t}(0, x) = v_0(x)$.
\end{itemize}

\section*{4. Teorema di Plancherel}
La trasformata di Fourier:
\begin{align*}
    \mathcal{F}: L^2(\mathbb{R}^d) \to L^2(\mathbb{R}^d),
\end{align*}
è un'isometria e un'operatore unitario:
\begin{itemize}
    \item $\|f\|_{L^2} = \|\hat{f}\|_{L^2}$.
    \item È suriettiva.
\end{itemize}

\section*{5. Conclusione}
L'analisi armonica è la branca dell'analisi matematica che studia la rappresentazione di funzioni e segnali come sovrapposizioni di onde fondamentali (armoniche).



\section*{I. Verso la meccanica ondulatoria}
Nel 1912, in Austria, si assiste a uno sviluppo significativo negli studi matematici e fisici, con particolare interesse verso la natura ondulatoria della materia e della luce.

\section*{2. Meccanica classica}
La meccanica classica pone le basi per la comprensione dei fenomeni fisici, ma presenta limiti nell'interpretare fenomeni atomici e subatomici. 

\textbf{Elettromagnetismo}: grazie agli studi di Maxwell, Hertz e Lorentz, si stabilisce che la luce è un'onda elettromagnetica. Hertz dimostra sperimentalmente che una particella carica in moto accelerato emette radiazione elettromagnetica.

\subsection*{Problemi aperti}
Nonostante i progressi, restano misteri irrisolti:
\begin{itemize}
    \item Gli spettri atomici, che mostrano righe caratteristiche per ogni elemento.
    \item La struttura atomica, la cui stabilità e organizzazione non sono spiegate dalla meccanica classica.
\end{itemize}

\section*{Mistero degli spettri atomici}
Il termine \emph{spettro}, dal latino \emph{imago} (immagine, fantasma), si riferisce alla distribuzione delle lunghezze d'onda della luce emessa o assorbita dagli atomi.

\textbf{Newton (1666)} osserva la scomposizione della luce bianca attraverso un prisma, descrivendo il fenomeno nella sua opera "Opticks" (1714).

Se al posto di luce bianca si utilizza un tubo contenente gas, come l'idrogeno, si osserva che lo spettro non è continuo ma presenta righe isolate. Questa scoperta porta Fraunhofer (1814) a inventare lo spettroscopio, uno strumento per analizzare gli spettri luminosi.

\subsection*{Ipotesi di Bunsen e Kirchhoff}
Nel 1859, Robert Bunsen e Gustav Kirchhoff formularono ipotesi fondamentali per la comprensione delle righe spettrali. Essi scoprirono che ogni elemento chimico emette luce a lunghezze d'onda specifiche quando viene riscaldato, producendo uno spettro caratteristico di righe luminose. Questo fenomeno è alla base della spettroscopia, una tecnica analitica che permette di identificare gli elementi presenti in una sostanza.

\textbf{Leggi di Kirchhoff}
Kirchhoff formulò tre leggi fondamentali della spettroscopia:
\begin{enumerate}
    \item \textbf{Prima legge}: Un corpo solido, liquido o un gas denso riscaldato emette uno spettro continuo di radiazione.
    \item \textbf{Seconda legge}: Un gas rarefatto riscaldato emette uno spettro di emissione costituito da righe luminose su uno sfondo scuro. Le posizioni delle righe sono caratteristiche dell'elemento chimico presente nel gas.
    \item \textbf{Terza legge}: Un gas freddo posto davanti a una sorgente di luce che emette uno spettro continuo produce uno spettro di assorbimento. Questo spettro è costituito da righe scure che corrispondono esattamente alle posizioni delle righe di emissione del gas.
\end{enumerate}

\textbf{Spettroscopia di Emissione e Assorbimento}
La spettroscopia di emissione e assorbimento si basa sulle leggi di Kirchhoff. Quando un elemento viene riscaldato, gli elettroni nei suoi atomi assorbono energia e saltano a livelli energetici superiori. Quando gli elettroni ritornano ai livelli energetici inferiori, emettono energia sotto forma di luce a lunghezze d'onda specifiche, producendo uno spettro di emissione.

Al contrario, quando la luce bianca passa attraverso un gas freddo, gli elettroni negli atomi del gas assorbono energia a lunghezze d'onda specifiche, saltando a livelli energetici superiori. Questo produce uno spettro di assorbimento con righe scure nelle stesse posizioni delle righe luminose dello spettro di emissione.

\textbf{Importanza delle Scoperte di Bunsen e Kirchhoff}
Le scoperte di Bunsen e Kirchhoff hanno avuto un impatto significativo sulla chimica e sull'astronomia. La spettroscopia è diventata uno strumento essenziale per l'analisi chimica, permettendo di identificare gli elementi presenti in campioni sconosciuti. In astronomia, la spettroscopia ha permesso di determinare la composizione chimica delle stelle e delle galassie, fornendo informazioni cruciali sulla loro formazione ed evoluzione.

Le ipotesi di Bunsen e Kirchhoff hanno gettato le basi per lo sviluppo della spettroscopia moderna e hanno contribuito in modo significativo alla comprensione della natura della luce e della materia.

\subsubsection*{Serie di Balmer}
Johann Balmer, nel 1885, scopre una formula empirica che descrive le lunghezze d'onda delle righe spettrali visibili dell'idrogeno. La serie di Balmer è data da:
\[
\frac{1}{\lambda} = R_H \left( \frac{1}{2^2} - \frac{1}{n^2} \right)
\]
dove \( \lambda \) è la lunghezza d'onda della luce emessa, \( R_H \) è la costante di Rydberg per l'idrogeno, e \( n \) è un numero intero maggiore di 2 (n = 3, 4, 5, ...).

Le righe della serie di Balmer si trovano nella regione visibile dello spettro elettromagnetico e sono particolarmente importanti perché furono le prime a essere osservate e studiate. Le prime quattro righe della serie di Balmer sono note come H-alfa, H-beta, H-gamma e H-delta, corrispondenti a transizioni verso il secondo livello energetico (n=2) da livelli superiori (n=3, 4, 5, 6).

\subsubsection*{Formula di Rydberg}
Johannes Rydberg generalizza la formula di Balmer per descrivere tutte le serie spettrali dell'idrogeno. La formula di Rydberg è:
\[
\frac{1}{\lambda} = R_H \left( \frac{1}{n_1^2} - \frac{1}{n_2^2} \right)
\]
dove \( n_1 \) e \( n_2 \) sono numeri interi tali che \( n_2 > n_1 \). Questa formula permette di calcolare le lunghezze d'onda delle transizioni elettroniche tra diversi livelli energetici dell'atomo di idrogeno.

La formula di Rydberg può essere utilizzata per descrivere altre serie spettrali dell'idrogeno, come la serie di Lyman (ultravioletto), la serie di Paschen (infrarosso), la serie di Brackett e la serie di Pfund. Ogni serie è caratterizzata da un valore specifico di \( n_1 \):
\begin{itemize}
    \item Serie di Lyman: \( n_1 = 1 \)
    \item Serie di Balmer: \( n_1 = 2 \)
    \item Serie di Paschen: \( n_1 = 3 \)
    \item Serie di Brackett: \( n_1 = 4 \)
    \item Serie di Pfund: \( n_1 = 5 \)
\end{itemize}

La scoperta delle serie spettrali e la formulazione della legge di Rydberg furono fondamentali per lo sviluppo della teoria quantistica e per la comprensione della struttura atomica.
La serie di Balmer si verifica quando un elettrone in un atomo di idrogeno cade da un livello energetico superiore a \( n = 2 \). Questo avviene perché gli elettroni negli atomi possono occupare solo livelli energetici discreti. Quando un elettrone assorbe energia, può saltare a un livello energetico superiore. Successivamente, quando l'elettrone ritorna a un livello energetico inferiore, emette energia sotto forma di luce. 

Nel caso della serie di Balmer, l'elettrone cade verso il secondo livello energetico (\( n = 2 \)) da livelli superiori (\( n > 2 \)). Le transizioni che coinvolgono il secondo livello energetico producono righe spettrali nella regione visibile dello spettro elettromagnetico. Questo è il motivo per cui la serie di Balmer è particolarmente importante e fu una delle prime ad essere osservata e studiata.

Le righe della serie di Balmer sono visibili perché la differenza di energia tra i livelli coinvolti corrisponde a lunghezze d'onda che rientrano nella gamma della luce visibile. Questo rende la serie di Balmer fondamentale per la spettroscopia e per la comprensione della struttura atomica.

\section*{Mistero della struttura atomica}
\textbf{Ludwig Boltzmann e Ernst Mach}: sostengono l'ipotesi atomica e l'energetismo. La materia può essere descritta come composta da unità discrete (atomi o molecole).

La stechiometria introduce il concetto di mole come unità della materia. L'equazione di Boltzmann diventa la base per le teorie cinetiche dei gas.

\section*{Modelli atomici del 1900}

Nel corso del 1900, la comprensione della struttura atomica ha subito una serie di evoluzioni significative grazie ai contributi di diversi scienziati. Tra i modelli più influenti vi sono quelli proposti da J.J. Thomson, Ernest Rutherford e Niels Bohr.

\subsection*{Modello di Thomson}
Nel 1897, J.J. Thomson scoprì l'elettrone, una particella subatomica carica negativamente. Basandosi su questa scoperta, Thomson propose il cosiddetto "modello a panettone" o "modello a cocomero" dell'atomo. In questo modello, gli elettroni erano dispersi all'interno di una sfera carica positivamente, simile ai semi di un cocomero nella polpa. Questo modello spiegava la neutralità elettrica dell'atomo, ma non riusciva a spiegare i dettagli degli spettri atomici.

\subsection*{Modello di Rutherford}
Nel 1911, Ernest Rutherford, attraverso il famoso esperimento della lamina d'oro, propose un nuovo modello atomico. Rutherford scoprì che la maggior parte delle particelle alfa passava attraverso la lamina d'oro senza deviazioni significative, ma alcune venivano deviate con angoli elevati e alcune addirittura rimbalzavano indietro. Da queste osservazioni, Rutherford dedusse che la carica positiva dell'atomo e la maggior parte della sua massa erano concentrate in un piccolo nucleo centrale, mentre gli elettroni orbitavano attorno a questo nucleo. Questo modello, tuttavia, non spiegava la stabilità degli atomi né gli spettri di emissione.
Il modello di Rutherford implica che gli elettroni dovrebbero spiraleggiare verso il nucleo in tempi brevissimi (\(\sim 10^{-10}\) s) per via della legge di Hertz. Questo contrasta con la stabilità osservata della materia.


\subsection*{Modello di Bohr}
Nel 1913, Niels Bohr propose un modello atomico che combinava le idee di Rutherford con i concetti della teoria quantistica. Bohr suggerì che gli elettroni orbitano attorno al nucleo in orbite stazionarie senza emettere energia. Gli elettroni possono saltare da un'orbita all'altra emettendo o assorbendo quanti di energia, con la frequenza della radiazione emessa data dalla differenza di energia tra le orbite divisa per la costante di Planck. Questo modello spiegava con successo gli spettri di emissione dell'idrogeno e introduceva il concetto di quantizzazione dell'energia negli atomi.
Il modello di Bohr rappresenta un passo cruciale nella comprensione della struttura atomica, introducendo concetti che sarebbero stati fondamentali per la meccanica quantistica. Ecco alcuni dettagli chiave del modello di Bohr:

\subsubsection*{Postulati di Bohr}
Bohr formulò tre postulati principali per descrivere il comportamento degli elettroni negli atomi:
\begin{enumerate}
    \item \textbf{Orbite stazionarie}: Gli elettroni orbitano attorno al nucleo in orbite stazionarie senza emettere radiazione elettromagnetica. Queste orbite sono quantizzate, cioè solo certe orbite con specifiche energie sono permesse.
    \item \textbf{Quantizzazione del momento angolare}: Il momento angolare dell'elettrone in un'orbita stazionaria è quantizzato ed è dato da \( L = n\hbar \), dove \( n \) è un numero intero positivo (numero quantico principale) e \( \hbar \) è la costante di Planck ridotta.
    \item \textbf{Transizioni quantiche}: Gli elettroni possono saltare da un'orbita stazionaria a un'altra solo assorbendo o emettendo un quanto di energia \( E = h\nu \), dove \( h \) è la costante di Planck e \( \nu \) è la frequenza della radiazione emessa o assorbita.
\end{enumerate}

\subsubsection*{Energia degli Elettroni}
L'energia totale di un elettrone in un'orbita stazionaria è data dalla somma dell'energia cinetica e dell'energia potenziale. Per l'atomo di idrogeno, l'energia dell'elettrone in un'orbita con numero quantico principale \( n \) è:
\[
E_n = - \frac{13.6 \text{ eV}}{n^2}
\]
dove \( 13.6 \text{ eV} \) è l'energia di ionizzazione dell'idrogeno. Questa formula mostra che l'energia degli elettroni negli atomi è quantizzata e diminuisce all'aumentare del numero quantico principale \( n \).
Il modello di Bohr spiega con successo gli spettri di emissione e assorbimento dell'idrogeno predetti dalla formula di Rydberg. 

\subsubsection*{Limitazioni del Modello di Bohr}
Nonostante il successo nel descrivere l'atomo di idrogeno, il modello di Bohr presenta alcune limitazioni:
\begin{itemize}
    \item Non riesce a spiegare gli spettri di atomi più complessi dell'idrogeno.
    \item Non tiene conto delle interazioni tra elettroni in atomi con più di un elettrone.
    \item Non spiega la struttura fine degli spettri atomici, che richiede una descrizione relativistica degli elettroni.
\end{itemize}
Il modello di Bohr ha avuto un impatto duraturo sulla fisica, ponendo le basi per lo sviluppo della meccanica quantistica. I concetti di quantizzazione dell'energia e delle orbite stazionarie sono stati fondamentali per la successiva teoria quantistica degli atomi, sviluppata da Schrödinger, Heisenberg e altri.

Svilupperemo ora i calcoli di Bohr per l'atomo di idrogeno, che portano alla formula di Rydberg e alla spiegazione della serie di Balmer. Partiamo dalla quantizzazione dei raggi delle orbite stazionarie nell'idrogeno. La legge di Newton in sola presenza di forza elettrostatica diventa: (ponendo le costanti dielettriche ad 1) 
$$m\frac{v^2}{r} = \frac{e^2}{r^2}$$e poichè $$L = mvr = n\hbar$$ otteniamo $$v = \frac{e^2}{n\hbar}$$ e sostituendo in $$m\frac{v^2}{r} = \frac{e^2}{r^2}$$ otteniamo $$r = \frac{n^2\hbar^2}{me^2}$$ che è la quantizzazione dei raggi delle orbite stazionarie. Sostituendo in ricordando che $$E_{tot}= E_{pot} + E_{cin} = -\frac{e^2}{2r} +  m \dfrac{v^2}{2} = -\frac{e^2}{r} + \frac{e^2}{2r} = -\frac{e^2}{2r}$$ otteniamo $$E_n = -\frac{e^2}{2r} = -\frac{e^2}{2} \frac{me^2}{n^2\hbar^2} = -\frac{me^4}{2n^2\hbar^2}$$ che è l'energia delle orbite stazionarie. Infine, la differenza di energia tra due orbite stazionarie è data da $$\Delta E = E_{n_2} - E_{n_1} = \frac{me^4}{2\hbar^2} \left( \frac{1}{n_1^2} - \frac{1}{n_2^2} \right)$$ che è la formula di Rydberg, infine ricordando semplicemente l'equazione di De Broglie $E = h \ni$ otteniamo le varie frequenze emesse nei salti energetici. 


D'ora in poi nel resto delle note seguiremo la seguente notazione:
\begin{itemize}
    \item \textbf{Spazio delle posizioni}: $\mathbb{R}^d \simeq E^d \ni x,\, y,\,z$ con $[x] = [y] = [z] = L \text{ lunghezza}$.
    \item \textbf{Spazio dei vettori d'onda}: $\mom\simeq \hat{E}^d \ni k, \, \xi, \, \alpha , \, \beta$ con $[k] = [ \xi ] = [\alpha] = [\beta] = L^{-1} \text{ inverso della lunghezza}$.
\end{itemize}

\begin{definition}
    Sia $f \in L^1(\mathbb{R}^d)$, chiameremo trasformata di Fourier e antitrasformata di Fourier le mappe lineari da $L^1(\mathbb{R}^d) $ a $ L^\infty(\mathbb{R}^d)$ definite rispettivamente da: 
    \begin{align*}
      (\mathcal{F} f)(k) &= \hat{f}(k) = \dfrac{1}{(2 \pi)^\frac{d}{2}}\int_{\mathbb{R}^d} e^{-i k \cdot x} f(x)  dx \\
    (\mathcal{F}_- f)(x) &= \dfrac{1}{(2 \pi)^\frac{d}{2}}\int_{\mathbb{R}^d} e^{i k \cdot x} f(k)  dk \\
    \end{align*}

\end{definition}

Definiremo le seguenti applicazioni, dati $a \in \mathbb{R}^d$,  $\alpha \in \mom$ e $\lambda \in \mathbb{R}$:
\begin{itemize}
    \item \textbf{Traslazioni}: $(\tau_a f)(x) = f(x - a)$.
    \item \textbf{Moltiplicazione per carattere}: $(c_\alpha f)(x) = e^{i \alpha \cdot k} f(x)$.
    \item \textbf{Dilatazioni/Contrazioni}: $(D_\lambda f)(x) = \lambda^{-d/2} f(\dfrac{x}{\lambda})$.
\end{itemize}

\begin{proposition}
 La formula sopra definisce una mappa lineare continua $F: L^1(\mathbb{R}^d) \to C_b(\mathbb{R}^d) \cap L^\infty(\mathbb{R}^d)$ tale che:
\begin{itemize}
    \item $\mathcal{F}(\tau_a f)(k) = e^{-i k \cdot a} \hat{f}(k)$.
    \item $\mathcal{F}(c_\alpha f)(k) = \hat{f}(k - x)$.
    \item $\mathcal{F}(D_\lambda f)(k) = \lambda^{d/2} \hat{f}(a k)= D_\frac{1}{\lambda} \hat{f}(k)$.
\end{itemize}
\end{proposition}

\begin{proof} *
    La dimostrazione è abbastanza semplice, per dimostrare che è in $C_b$ basta vedere che $\| \hat{f}\|_\infty \leq \| f\|_1$ e per dimostrare che è continua basta vedere che data $\xi_j \to \xi$ convergente allora $e^{-i \xi_j \cdot x} f(x) \to e^{-i \xi \cdot x} f(x)$ quasi ovunque e inotre $|e^{-i \xi_j \cdot x} f(x)| \leq |f(x)| \in L^1(\mathbb{R}^d)$, il resto segue facendo i conti.
\end{proof}


\begin{proposition}[Lemma di Rienmann-Lebesgue]
    Sia $f \in L^1(\mathbb{R}^d)$, allora $\hat{f}(k)$ tende a $0$ per $|k| \to \infty$.
\end{proposition}



\section*{2. Estensione della teoria su $L^2(\mathbb{R}^d)$}
Il problema è che nella teoria in $L^1$ l'inversa della trasformata di Fourier non esiste, per ovviare a questo problema prima si studia la restrizione della trasformata allo spazio di Schwarts $\Sw$ dove trasformata e antitrasformata di Fourier sono una l'inversa dell'altra si passa alle classi di equivalenza e infine si estende per linearità e continuità a $L^2(\mathbb{R}^d)$, il che fa si, poichè $\mathcal{F}\mathcal{F}_- = \un_{\Sw}$ e che l'estensione dell'indentià è unica vale anche $\mathcal{F}\mathcal{F}_- = \un_{L^2}$. In maniera formale avremo che:
\begin{theorem}
    Lo spazio di Schwartz è invariante sotto l'azione della trasfromata e l'antitrasformata di Fourier, le due trasformazioni ristrette a $\Sw$ sono una l'inversa dell'altra e sono isometrie per il prodotto scalare di $L^2(\mathbb{R}^d)$. \label{thm:Fourier}
\end{theorem}

La seguente dimostrazione è presa da \cite{Mor}

\begin{proof} *     
Dimostriamo ora per $\mathcal{F}$, il risultato per $\mathcal{F}_{-}$ è analogo.

È immediato verificare che:
\begin{equation*}
    |\partial_k^\alpha e^{i k \cdot x} f(x)| = |i^{|\alpha|} M_\alpha(x) f(x)| \leq |M_\alpha(x) f(x)|,
\end{equation*}
con $M_\alpha(x) = x^\alpha f(x)$ l'operatore moltiplicativo. 

Poiché $f \in S(\mathbb{R}^n)$, segue che $M_\alpha(x) f(x) \in L^1(\mathbb{R}^n)$. Usando il teorema della convergenza dominata di Lebesgue, possiamo scambiare derivata e integrale:
\begin{equation*}
    \partial_k^\alpha \hat{f}(k) = i^{|\alpha|} \int_{\mathbb{R}^n} e^{i k \cdot x} \frac{1}{(2\pi)^{n/2}} M_\alpha(x) f(x) dx.
\end{equation*}


Notando che $f$ si annulla più rapidamente di ogni potenza inversa di $|x|$ per $|x| \to +\infty$, si trova che:
\begin{equation*}
M_{{\beta}}({k}) \hat{f}(k) = \int_{\mathbb{R}^n} (-i)^{|{\beta}|} \partial^{{\beta}}_{{x}} \left( \frac{e^{i{k} \cdot {x}}}{(2\pi)^{n/2}} \right) f({x}) \, \dd {x},
\end{equation*}
e quindi, usando l'integrazione per parti:
\begin{equation*}
M_{{\beta}}({k}) \hat{f}(k) = i^{|{\beta}|} \int_{\mathbb{R}^n} \frac{e^{i{k} \cdot {x}}}{(2\pi)^{n/2}} \partial^{{\beta}}_{{x}} f({x}) \, \dd {x}.
\end{equation*}

Quindi, inserendo $\partial^{{\alpha}}_{{k}} g$ al posto della funzione $g$ in (3.64) e tenendo conto di (a), si ha:
\begin{equation*}
|M_{{\beta}}({k}) \partial^{{\alpha}}_{{k}} \hat{f}(k)| \leq \left| \partial^{{\beta}} (M_{{\alpha}} f) \right|_1,
\end{equation*}
per ogni ${k} \in \mathbb{R}^n$. Essendo finito il secondo membro, in quanto $f \in \mathcal{S}(\mathbb{R}^n)$, ed essendo ${\alpha}$ e ${\beta}$ arbitrari, concludiamo che $\hat{f} \in \mathcal{S}(\mathbb{R}^n)$.

Possiamo riscrivere le equazioni sopra come:
\begin{align*}
\partial^{{\alpha}} \mathcal{F} &= i^{|{\alpha}|} \mathcal{F} M_{{\alpha}},  \\
M_{{\beta}} \mathcal{F} &= i^{|{\beta}|} \mathcal{F} \partial^{{\beta}}, 
\end{align*}
dove $\mathcal{F}$ è in realtà la restrizione della trasformata di Fourier a $\mathcal{S}(\mathbb{R}^n)$. Osservando che:
\begin{equation*}
\mathcal{F} h = \mathcal{F}_{-} h \quad \text{per ogni } h \in \mathcal{S}(\mathbb{R}^n),
\end{equation*}
si ricava facilmente:
\begin{align*}
\partial^{{\alpha}} \mathcal{F}_{-} &= (-1)^{|{\alpha}|} i^{|{\alpha}|} \mathcal{F}_{-} M_{{\alpha}},  \\
M_{{\beta}} \mathcal{F}_{-} &= (-1)^{|{\beta}|} i^{|{\beta}|} \mathcal{F}_{-} \partial^{{\beta}}. 
\end{align*}

Abbiamo dimostrato in particolare che valgono le seguenti:
\begin{align*}
\mathcal{F}\mathcal{F}_{-} M_{{\alpha}} &= M_{{\alpha}} \mathcal{F}\mathcal{F}_{-},  \\
\mathcal{F}_{-}\mathcal{F} M_{{\alpha}} &= M_{{\alpha}} \mathcal{F}_{-}\mathcal{F}, 
\end{align*}
 e anche
\begin{align*}
\mathcal{F}\mathcal{F}_{-} \partial^{{\alpha}} &= \partial^{{\alpha}} \mathcal{F}\mathcal{F}_{-},  \\
\mathcal{F}_{-}\mathcal{F} \partial^{{\alpha}} &= \partial^{{\alpha}} \mathcal{F}_{-}\mathcal{F}. 
\end{align*}

Mostreremo ora che, in virtù di tali relazioni di commutazione, gli operatori
\begin{equation*}
J := \mathcal{F}\mathcal{F}_{-} \quad \text{e} \quad J_{-} := \mathcal{F}_{-}\mathcal{F}
\end{equation*}
devono essere l'operatore identità di $\mathcal{S}(\mathbb{R}^n)$. Per prima cosa proviamo che, fissati ${x}_0 \in \mathbb{R}^n$ e $f \in \mathcal{S}(\mathbb{R}^n)$, il valore di $(Jf)({x}_0)$ dipende solo da $f({x}_0)$. Se $f \in \mathcal{S}(\mathbb{R}^n)$ possiamo sempre scrivere:
\begin{equation*}
f({x}) = f({x}_0) + \int_0^1 \dv{f}{t}({x}_0 + t({x} - {x}_0)) \dd{t} = f({x}_0) + \sum_{i=1}^n (x_i - x_{0i}) g_i({x}),
\end{equation*}
dove le funzioni $g_i$ (che sono $C^\infty(\mathbb{R}^n)$, come si verifica facilmente) sono definite da:
\begin{equation*}
g_i({x}) := \pdv{}{x_i} \int_0^1 f({x}_0 + t({x} - {x}_0)) \dd{t}.
\end{equation*}

Applicando $J$ ad ambo i membri di questa decomposizione e tenendo conto del fatto che $J$ commuta con i polinomi in ${x}$ per (3.69), otteniamo:
\begin{equation*}
(Jf_1)({x}) = (Jf_2)({x}) + \sum_{i=1}^n (x_i - x_{0i}) (Jg_i)({x}).
\end{equation*}
Prendendo ${x} = {x}_0$, si vede che $(Jf_1)({x}_0) = (Jf_2)({x}_0)$ sotto l'ipotesi iniziale $f_1({x}_0) = f_2({x}_0)$. Quindi, come detto, $(Jf)({x}_0)$ è una funzione soltanto di $f({x}_0)$. Tale funzione deve essere anche lineare, dato che $J$ è lineare per costruzione. Ne consegue che sarà:
\begin{equation*}
(Jf)({x}_0) = j({x}_0) f({x}_0),
\end{equation*}
dove $j$ è una funzione su $\mathbb{R}^n$ a valori in $\mathbb{C}$. Poichè $x_0$ è arbitrario allora abbiamo provato che $J$ agisce come la moltiplicazione per una funzione $j$. È poi immediato verificare che tutte le derivate di $j$ sono nulle e calcolando la trasformata di Fourier della gaussiana si ottiene che tale costante è proprio $1$.

\end{proof}

\begin{theorem}[Estensione limitata]
    Siano $E_1, E_2$ spazi di Banach, $D$ un sottoinsieme denso di $E_1$ e $T: D \to E_2$ un operatore lineare prelimitato(ossia $\|T\psi\| \leq c \|\psi\|$ per ogni $\psi \in D$). Allora esiste un'unica estensione lineare e continua $\tilde{T}: E_1 \to E_2$ tale che $\|\tilde{T}\psi\| \leq c \|\psi\|$ 
\end{theorem}

La dimostrazione di questo teorema è nota. Ma ci permette ora, sapendo che $\|\mathcal{F}_{\Sw} \psi \|_{L^2} = \|\psi\|_{L^2}$ per ogni $\psi \in \Sw$ di estendere la trasformata di Fourier a tutto $L^2(\mathbb{R}^d)$ e di estendere la sua inversa, l'antitrasformata, su $\Sw$ all'inversa su $L^2(\mathbb{R}^d)$.

\begin{theorem}[Fourier-Plancherel]
    La trasformata di Fourier si estende per linearità e continutià ad una mappa lineare $\mathcal{F} : L^2(\mathbb{R}^d) \to L^2(\mathbb{R}^d)$ tale che: 
\begin{enumerate}
    \item Preserva il prodotto hermitiano.
    \item È una isometria.
    \item È suriettiva.
\end{enumerate}
\end{theorem}
\begin{proof}* 
    La dimostrazione è immediata dal Teorema \ref{thm:Fourier} e dai commenti ad inizio sezione.
\end{proof}

\section*{3. Ottica ondulatoria ed equazione delle onde}

\begin{definition}
Chiameremo equazione delle onde il sistema: 
    \begin{align*}
        \Box u = \frac{\partial^2 u}{\partial t^2} - \Delta u = 0,
    \end{align*}
    con condizioni iniziali $u(0, x) = u_0(x)$, $\frac{\partial u}{\partial t}(0, x) = v_0(x)$.
Incognita: $u: \mathbb{R}\times \mathbb{R}^d \to \mathbb{C}$ tale che $(t,x) \mapsto u(t, x)$.
\end{definition}

Una \emph{soluzione} dell'equazione delle onde è una funzione $u: \mathbb{R} \to L^2(\mathbb{R}^d) = \hi$ che a $t$ associa $u(t, \cdot)$. Scegliamo $u(t, \cdot)$ a valori in $L^2(\mathbb{R}^d)$ perchè è uno spazio di Hilbert ed è dotato di na buona teoria di Fouier. Una buona soluzione deve essere tale che: 
\begin{itemize}
    \item Deve avere regolarità nel tempo: $$ \partial_t u(t, \cdot) = L^2 -\lim_{h \to 0} \frac{u(t+h, \cdot) - u(t, \cdot)}{h}$$ se ipotizziamo una funzione $g$ come derivata prima dobbiamo controllare che $$ \| \frac{u(t+h , \cdot )- u (t,\cdot)}{h} - g \|_{L^2} \to 0 $$, si parla quindi di derivabilità quasi ovunque.
    \item $\dfrac{1}{c^2}\dfrac{\partial^2 u}{\partial t^2} = \Delta u$ notiamo subito che non è sempre vero che $\Delta u \in L^2$ introduciamo quindi lo spazio $\hi^s(\mathbb{R}^d):= \{ f \in \Lr : \int_{\mom}(1 + |k|^2)^s |\hat{f}(k)|^2 dk < \infty \}$ con $s \in \mathbb{R}$ è possibile dimostrare che la $s$-esima derivata distribuzionale è integrabile se e solo se $f \in \hi^s(\mathbb{R}^d)$. Questi spazi sono inscatolati tra loro e sono tutti densi in $L^2(\mathbb{R}^d)$, vale quindi $\Lr \supseteq \hi^0(\mathbb{R}^d) \supseteq \hi^1(\mathbb{R}^d) \supseteq \hi^2(\mathbb{R}^d) \supseteq \ldots$.
    
\end{itemize}





\include{Chap3}
\include{Chap2}
\chapter{Monaco}

\begin{definition}
    Sia $T$ operatore nello spazio di Hilbert $\hi$:
\begin{enumerate}
    \item Chiameremo insieme risolvente di $T$ l'insieme $\rho(T)$ dei numeri complessi $z$ tali che:
    \begin{itemize}
        \item $T-z\un$ è iniettivo.
        \item $\Ran(T- z \un)$ è denso in $\hi$.
        \item $(T-z\un)^{-1}$ è limitato.
    \end{itemize}
    \item Chiameremo risolvente di $T$ l'operatore $(T-z\un)^{-1}$ per $z \in \rho(T)$.
    \item Chiameremo spettro di $T$ l'insieme $\sigma(T) := \mathbb{C} \setminus \rho(T)$. Lo spettro è unione disgiunta di tre insiemi:
    \begin{itemize}
        \item Spettro puntuale $\sigma_p(T)$ formato dai $\lambda \in \mathbb{C}$ tali che $T-\lambda \un$ non è iniettivo.
        \item Spettro continuo $\sigma_c(T)$ formato dai $\lambda \in \mathbb{C}$ tali che $T-\lambda \un$ è iniettivo, $\Ran(T-\lambda \un)$ è denso in $\hi$ ma $(T-\lambda \un)^{-1}$ non è limitato.
        \item Spettro residuale $\sigma_c(T)$ formato dai $\lambda \in \mathbb{C}$ tali che $T-\lambda \un$ è iniettivo e $\Ran(T-\lambda \un)$ non è denso in $\hi$.
    \end{itemize}
\end{enumerate}
\end{definition}


\section{Sistemi invarianti per rotazioni}


\section*{Modellizzazione matematica dell'atomo di idrogeno}
Studieremo la dinamica di una particella quantistica carica $e^-$ sotto l'azione del campo di forze elettrostatiche generato dal nucleo: 

\begin{equation*}
-\nabla V(\mathbf{r}) = \mathbf{F}(\mathbf{r}) = \dfrac{\gamma}{|\mathbf{r}|^3}\mathbf{r} , \quad V(\mathbf{r}) = \frac{-e^2}{|\mathbf{r}|}
\end{equation*}

Il modello quantistico ha per hamiltoniana:
\begin{equation*}
H = H_0 + V = \frac{\mathbf{p}^2}{2m} + V(\mathbf{r})
\end{equation*}
Più in generale, $H = H_0 + V$, che sono sistemi invarianti per rotazioni quindi 
\begin{equation*}
G = SO(3) = \{R \in M_{3 \times 3}(\mathbb{R}) : R^T R = I, \det R = 1\}
\end{equation*}
è il gruppo di simmetrie dinamiche del sistema.

Per descrivere $SO(3)$ possiamo specificare:
\begin{itemize}
    \item Asse di rotazione $\hat{n} \in S^2$;
    \item Angolo di rotazione $\alpha \in [\pi, \pi]$.
\end{itemize}

Prendere un tale $\alpha$ in tale dominio è ridondate poichè $R(\alpha, \hat{n}) = R(-\alpha, -\hat{n})$ ma tale scelta ci sarà utile in seguito.

Esempio:

\begin{equation*}
    R(\alpha, \hat{e_3}) = \begin{pmatrix}
    \cos \alpha & \sin \alpha & 0 \\           
    -\sin \alpha & \cos \alpha & 0 \\
    0 & 0 & 1   
\end{pmatrix}
\end{equation*}

Per questa lezione lavoreremo nel costesto $\hi = L^2(\mathbb{R}^3)$ facendo agire $SO(3)$ sul sistema quantistico come:
\begin{equation*}
   SO(3) \to \mathcal{U}(\hi) \quad R \mapsto (U_R \psi)(\mathbf{r}) := \psi(R^{-1}\mathbf{r})
\end{equation*}

Diciamo che $H$ descrive un sistema quantistico invariante per rotazioni se vale $U^{-1}_R H U_R = H$ per ogni $R \in SO(3)$ dal teorema di Noether $U^{-1}_R H U_R = H$ se e solo se $[H, L_i] = 0$ dove $L_i$ sono i generatori infinitesimali di $SO(3)$.

Da Stone:
\begin{equation*}
U(\alpha, \hat{e_i}) = e^{\frac{-i\alpha L_i}{\hbar} }, \quad L_i=L_i^*, \mathbf{L}=(L_1, L_2, L_3)
\end{equation*}
Dove $L_1, L_2, L_3$ sono gli operatori momento angolare, infatti 

\begin{proposition}
    $\mathbf{L}$ sono gli operatori momento angolare. Quindi $\mathbf{L}= \mathbf{x} \times \mathbf{p}$.
\end{proposition}

\begin{proof}
    Per brevità dimostreremo che $L_3 = x_1 p_2 - x_2 p_1 = -i \hbar (x_1 \partial_2 - x_2 \partial_1)$: %mettere a posto gli indici ora non mi va.
\end{proof}

\begin{proposition}
Gli operatori momento angolare soddisfano le regole di commutazione dell'algebra di Lie $\mathfrak{so}(3)$:
\begin{align*}
[L_1, L_2] &= i\hbar L_3, \\
[L_3, L_1] &= i\hbar L_2, \\
[L_2, L_3] &= i\hbar L_1.
\end{align*}
\end{proposition}
\begin{proof}
    Anche qui per brevità dimostreremo solo il primo commutatore:
\begin{align*}
    [L_1, L_2] =& [x_2 p_3 - x_3 p_2, x_3 p_1 - x_1 p_3] =\\
=& [x_2p_3, x_3p_1] - [x_2p_3, x_1p_3] - [x_3p_2, x_3p_1] + [x_3p_2, x_1p_3] =\\
=& i\hbar(x_1p_2 - x_2p_1) = i\hbar L_3
\end{align*} 
\end{proof}

Definiamo l'operatore di Casimir:
\begin{equation*}
\mathbf{L}^2 = L_x^2 + L_y^2 + L_z^2
\end{equation*}

\begin{proposition}
    L'operatore di Casimir commuta con ogni componente di $L$:
\begin{equation*}
[\mathbf{L}^2, L_i] = 0 \quad \forall i \in \{x, y, z\}
\end{equation*}
\end{proposition}

\begin{proof}
    Dimostriamo che $[\mathbf{L}^2, L_3] = 0$:

 \begin{align*}
    [\mathbf{L}^2, L_3] =& [L_1^2 + L_2^2 + L_3^2, L_3] = \\
  =& [L_1^2, L_3] + [L_2^2, L_3]= L_1[L_1, L_3] + [L_1, L_3]L_1 + L_2[L_2, L_3] + [L_2, L_3]L_2 = \\ 
=&  i\hbar L_1L_2 - i\hbar L_2L_1 - i\hbar L_1L_2 + i\hbar L_2L_1 = 0
\end{align*}
\end{proof}

Passiamo ora in coordinate sferiche per sfruttare la simmetria del sistema:
\begin{lemma}
    Esiste un isomorfismo unitario: 
\begin{align*}
\mathcal{R} :   L^2(\mathbb{R}_+) \Ht L^2(\mathbb{S}^2) \to& L^2(\mathbb{R}^3) \\ 
(r, \theta, \phi) \mapsto& f(r,\theta, \phi) := \dfrac{1}{r}R(r)Y(\theta, \phi)
\end{align*}
\end{lemma}

\begin{proof}
    \begin{align*}
    \|f\|^2=& \int_{\mathbb{R}^3} |f(\mathbf{x})|^2 \dd^3 x = \\
 =& \int_{\mathbb{R}_+ \times \mathbb{S}^2}r^2sin \theta \left| \frac{1}{r}R(r) Y (\theta, \phi)\right|^2 \dd r \dd \theta \dd \phi =\\ 
 =&  \int_{\mathbb{R}_+} |R(r)|^2 \dd r \int_{\mathbb{S}^2} |Y(\theta, \phi)|^2 \dd \theta \dd \phi =: \|R \otimes Y\|^2
    \end{align*}
 $f$ allora si estende ad un isomorfismo unitario. 
\end{proof}

%qui manca un pezzettino 

Ora per coloro che studiano matematica piccola digressione sul motivo formale per cui è interessante studiare l'operatore di Casimir, è immediato vedere come l'operatore di Casimir cosí definito sia un elemento del centro dell'algebra inviluppante universale di $\mathfrak{so}(3)\simeq \mathfrak{sl}(2)$ e quindi agirà come moltiplicazione per uno scalare sui moduli di Verma associati, ora è possibile dimostrare che è anche l'unico elemento del centro (per cui definisce univocamente il carattere associato al modulo di Verma) per cui per il teorema di Harish-Chandra basterà studiare la sua azione per conoscere completamente il modulo irriducibile associato. Per maggiori informazioni sulle rappresentazioni di $\mathfrak{so}(3)$ (che espande leggermente quello che vedremo tra poco) si veda \cite{Humphreys} Sezione 7, per una esaustiva trattaziona del teorema di Harish-Chandra, che esula dagli argomentri trattati in questo corso, si veda sempre \cite{Humphreys} Sezione 23. 


Per un sistema quantistico con simmetria rotazionale, lo spettro dell'operatore di Casimir è discreto:
\begin{equation*}
\mathbf{L}^2 Y_{\ell m} = \hbar^2 \ell(\ell+1) Y_{\ell m}
\end{equation*}
dove $\ell \in \mathbb{N}$ e $m \in \{-\ell, \dots, \ell\}$.

\section*{Conclusione}
La simmetria del gruppo $SO(3)$ permette di caratterizzare le proprietà dell'atomo di idrogeno e degli operatori associati.


\begin{definition}
    Un operatore Hermitiano $V$ è detto $H_0$-limitato se $\mathcal{D}(V) \supset \mathcal{D}(H_0)$ e esistono $C,D> 0$ tali che $\| V \psi \| \leq C \| H_0 \psi \| + D \| \psi \|$ per ogni $\psi \in \mathcal{D}(H_0)$. 
\end{definition}

È possibile dimostrare che $V$ è $H_0$-limitato se e solo se $\mathcal{D}(V) \subset \mathcal{D}(H_0)$ e $V(H_0 -z \un)^{-1}\in \B$ per ogni $z \in \rho(H_0)$. Definiamo inoltre il limite di $V$ rispetto a $H_0$ come $H_0\lim(V) = \sup_{z \in \rho(H_0)} \| V(H_0 -z \un)^{-1}\|$.

\begin{theorem}[Katô-Rellich]
    Sia $V$ un operatore hermitiano $H_0$-limitato con $H_0\lim(V)< 1$ allora $H = H_0 + V$ è autoaggiunto su $\mathcal{D}(H_0)$.
\end{theorem}

\begin{definition}
    Un operatore Hermitiano $V$ è detto $H_0$-compatto se $\mathcal{D}(V) \supset \mathcal{D}(H_0)$ e $V(H_0 -z \un)^{-1}\in \Bo$.
\end{definition}

Se $V$ è $H_0$-compatto allora $V$ è $H_0$-limitato e $H_0\lim(V) = 0$, inoltre $H_0 + V$ è essenzialmente autoaggiunto.

\begin{theorem}[Weyl]
    Perturbazioni $V$ $H_0$-compatte di un operatore autoaggiunto $H_0$ sono tali che $\sigma_{ess}(H_0 + V) = \sigma_{ess}(H_0)$.
\end{theorem}

\section{Stati puri e stati misti}
Chiameremo $\Bo$ l'insieme degli operatori compatti su $\hi$.

\begin{definition}
    Sia $\hi$ con la norma indotta $\| \dot \|$ dal prodotto scalare, diremo che $A \in \B$ è un operatore di Hilber-Schmidt se esiste una base hilbertiana $\{ u_k\}$ tale che $\sum \|Au_k \|<\infty$. 
\end{definition}

Indicheremo la classe di operatori di Hilber-Schmidt su $\hi$ come $\Hs$, è possibile inoltre dimostrare che $\Hs \subset \Bo$ è uno *-ideale bilatero.

\begin{proposition}
    Sia $\hi$ spazio di Hilbert, $T\in \B$. I seguenti tre fatti sono equivalenti:
\begin{enumerate}
    \item Esiste $N$ base hilbertiana tale che $\{ \langle u, |T| u \rangle \}$ ha somma finita.
    \item $\sqrt{T^*T}$ è di Hilbert-Schmidt.
    \item $T$ è compatto e la successione degli autovalori $\{m_n\}$ di $\sqrt{T^*T}$ contati con molteplicità ha somma finita. 
\end{enumerate}
\end{proposition}

\begin{definition}
    Sia $\hi$ spazio di Hilbert, $T\in \B$ è detto operatore classe traccia se vale una delle tre condizioni equivalenti della proposizione precedente. L'insieme degli operatori classe traccia su $\hi$ sarà indicato con $\Ct$, infine se $T \in \Ct$ allora definiamo $\| T\|_1 = \| \sqrt{T^*T}\|_2^2= \sum m_n$.
\end{definition}

Come per gli operatori di Hilbert-Schmidt $\Ct$ è uno *-ideale bilatero di $\B$. Inoltre $\Ct \subset \Hs \subset \Bo \subset \B$. Infine dato $A\in \Ct$ esistono $B, C \in \Hs$ tali che $A = BC$, viceversa se $B,C \in Hs$ allora $BC \in \Ct$.  
\begin{definition}
    Sia $\hi$ spazio di Hilbert, $T\in \Ct$, il numero $\mathbb{C} \ni \Tr T = \sum \langle u, Tu \rangle$ è detto traccia di $T$.
\end{definition}

Ora enunceremo un teorma con una serie di proprietà delle tracce, che non dimostreremo per brevità ma che si possono trovare in \cite{Mor}.

\begin{theorem}
    Sia $\hi$ spazio di Hilbert, $T \in \Ct$ allora valgono le seguenti:
\begin{itemize}
    \item $|T| \in \Ct$ e $\Tr |T| = \sum |m_n|=\|T\|_1$.
    \item Se $S \in \B$ allora $\Tr(ST) = \Tr(TS)$.
    \item Se $\hi$ è complesso allora $\Tr(T^*) = \overline{\Tr(T)}$.
    \item Se $T \in \Ct$ e $T = \sum T_i$ con $T_i \in \Ct$ allora $\Tr(T) = \sum \Tr(T_i)$.
    \item Se $\hi$ è complesso allora $\Tr(T) = \sum_{\lambda \in \sigma_p(T)} \lambda$, dove i $\lambda$ sono contati con molteplicità geometrica. 
\end{itemize}
\end{theorem}

La dimostrazione dell'ultimo punto è interessante ma piuttosto articolata e puó essere trovata in \cite{Bir}. 


\begin{definition}
    Una mappa lineare $\mathbb{E}: \B \to \mathbb{C}$ è detta \emph{ una famiglia di valori attesi} se valgono le seguenti:

\begin{itemize}
    \item $\mathbb{E}(\un) = 1$.
    \item $\mathbb{E}(A)$ è reale quando $A$ è autoaggiunto.
    \item $\mathbb{E}(A)$ è positivo quando $A$ è autoaggiunto e positivo.
    \item Per ogni successione $A_n \in \B$ se $\| A_n\psi - A \psi\| \to 0$ per tutti $\psi \in \hi$ allora $\Phi(A_n) \to \Phi(A)$.
\end{itemize}
\end{definition}

\begin{definition}
    Un operatore $\rho\in \Ct$ è una matrice densità se $\rho$ è autoaggiunto, non negativo e vale $\Tr \rho =1$.
\end{definition}

\begin{definition}
    Siano $\hi_1, \hi_2$ spazi di Hilber definiremo il \emph{prodotto tensore tra due spazi di Hilbert} $\hi_1 \hat{\otimes} \hi_2$ come il completamento di $\hi_1 \otimes \hi_2$ rispetto al prodotto:
$(u_1 \otimes v_1, u_2 \otimes v_2)= (u_1, u_2)_1 (v_1, v_2)_2$.
\end{definition}

È naturale definire il prodotto tensore tra operatori, siano $A_1 \in \mathfrak{B}(\hi_1)$ e $A_2 \in \mathfrak{B}(\hi_2)$ definiremo $A_1 \otimes A_2 \in \B(\hi_1 \hat{\otimes} \hi_2)$ come $A_1 \otimes A_2(u_1 \otimes u_2) = A_1 u_1 \otimes A_2 u_2$.
A questo punto sia $\rho$ matrice dentià su $\hi_1 \Ht \hi_2$ allora $\rho^{(1)}$ e $\rho^{(2)}$ sono le matrici densità ridotte su $\hi_1$ e $\hi_2$ rispettivamente definite come le uniche tali che $\tr(\rho(A \otimes \un))= \tr(\rho^{(1)}A)$ e $\tr(\rho(\un \otimes B))= \tr(\rho^{(2)}B)$ per ogni $A \in \mathfrak{B}(\hi_1)$ e $B \in \mathfrak{B}(\hi_2)$. Esistenza e unicità di tali matrici densità è dimostrata ad esempio in \cite{Hall} Theorem 19.13.


\begin{proposition}
    L'applicazione $ L^2(X_1, \mu_1) \Ht L^2(X_2, \mu_2) \to L^2(X_1 \times X_2, \mu_1 \times \mu_2) $ è un isomorfismo.
\end{proposition}

Per la dimostrazione di veda \cite{Hall}.

Se ho n sistemi quantistici composti descritti da $\hi_1, \dots, \hi_n$ allora il sistema composto è descritto dallo spazio di hilber $\hi = \hi_1 \Ht \dots \Ht \hi_n$. 
Per cui un sistema quantistico di n particelle puó essere descritto da $L^2(\mathbb{R}^{nd})= \Lr \Ht \dots \Ht \Lr$. Ora vogliamo generalizzare gli operatori a questi spazi tensore, 
siano $A_i^* = A_i\in \mathfrak{B}(\hi_1)$ definiremo l'osservabile del sistema composto come $A(\psi_1 \otimes \dots \otimes \psi_n)= A_1 \psi_1 \otimes \dots \otimes A_n \psi_n$.
In particolare:
\begin{enumerate}
    \item Se $A_i^* = A_i\in \mathfrak{B}(\hi_i)$ definisco $\B \ni A^{ (i)} := \un \otimes \un \otimes \dots \otimes \un \otimes A_i \otimes \un \otimes \dots \otimes \un$.
    \item Se $I = \{ i_1, \dots , i_k \} \subset \{1, \dots , n \}$ e $A \in \mathfrak{B}(\hi_{i_1}\Ht \dots \Ht \hi_{i_k})$ allora $A^{ (I)} \in \B$ agisce sui sottosistemi associati.
\end{enumerate}

\begin{theorem}
    Se $\rho^* = \rho \geq 0$ $\Tr(\rho) = 1$ matrice densità su $\hi$ allora siste una matrice densità $\rho^{(I)}$ su $\mathfrak{B}(\hi_{i_1}\Ht \dots \Ht \hi_{i_k})$ tale che $\mathbb{E}_\rho(A^{(I)}) = \mathbb{E}_{\rho^{(I)}}(A)$, con $\rho^{(I)}$ lo stato indotto da $\rho$ sul sottosistema $I$ chiamato matrice dentià ridotta e $\mathbb{E}_\rho^{(I)}$ è detta traccia parziale della famiglia $\mathbb{E}_\rho$.
\end{theorem}

\begin{proof}
    %Fare dopo
\end{proof}







\begin{thebibliography}{20}

\bibitem{Mor}
V. Moretti (2012) \emph{Teoria spettrale e meccanica quantistica}, Springer.

\bibitem{Hall}
Brian C. Hall (2013) \emph{Quantum Theory for Mathematicians}, Springer New York. 

\bibitem{Bir}
Birman, M.S., Solomjak, M.Z. (1987) \emph{Spectral Theory of Self-Adjoint Operators in Hilbert Space}, D. Reidel Publishing Company, Dordrecht.

\bibitem{Humphreys} 
James E. Humphreys (1972) \emph{Introduction to Lie Algebras and Representation Theory}, Springer.


\end{thebibliography}




\end{document}


