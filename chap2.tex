\chapter{Teoria spettrale}
\section{Autoaggiuntezza}

Partiamo con il toy model degli spazi di Hilbert finito dimensionali $\hi \simeq \mathbb{C}^n$ che ha, si una serie di utili applicazioni pratiche ad esempio si prenda la teoria dello spin, ma anche una teoria molto piú semplificata come vedremo.

\begin{theorem}
    Sia $H \in \End(\mathbb{C}^n)$ tale che $H^*= H$, valgono le seguenti proprietà
\begin{enumerate}
    \item $H$ genera un'evoluazione unitaria, ossia dato il sistema 
    \begin{equation*}
        \begin{cases}
            i \psi = H \psi \\
            \psi(0) = \psi_0
        \end{cases}
    \end{equation*}
    la cui soluzione è data da $\psi : \R \to \hi \simeq \mathbb{C}^n$ che associa $t \mapsto \psi(t):= e^{-itH}\psi_0$ con:
    $$e^{-itH}= \lim_{N \to \infty} \sum^N  \dfrac{1}{k!}(-itH)^k$$
    \item $H$ è ortodiagonizzabile: 
    \begin{enumerate}
        \item Esiste una base ortonormale ${e_1, \dots, e_n}$ di autovettori di $H$.
        \item Esiste $U : \hi \to \mathbb{C}^n$ tale che $U H U^{-1}$ è diagonale
        \item $H$ è spettrabizabile ossia esiste una famiglia $\{E_l\}$ di proiettori ortogonali mutualmente ortogonali tali che $\sum E_l = \un$ e $H = \sum \lambda_l E_l$.
    \end{enumerate}
\end{enumerate}
\end{theorem}
La dimostrazione è nota dall'algebra lineare.
Il problema nel caso infinito dimensionale è che per operatori hermitiani il teorema sopra è falso, si prenda ad esempio l'operatore posizione su $\R$, è simmetrico e quindi possiede solo autovalori reali ma è facile vedere che fissato $x_0 \in \R$ non esiste $\psi \in L^2(\R)$ non nulla tale che $X\psi = x_0\psi$ quindi non sempre esistono autovettori per operatori simmetrici nel caso infinito dimensionale.
Lo scopo di questa sezione sarà costruire le basi per generalizzarlo al caso infinito dimensionale e poi al caso di operatori non limitati.


\begin{definition}
    È detto \emph{grafico} di un operatore $T$ in $ \hi $ è il sottospazio di $ \hi \times \hi \supseteq \mathcal{G}(T):= \{ (x, Tx) | x \in \mathcal{D}(T) \}$ . 
Diremo inolte che $T$ è un operatore chiuso se il suo grafico è chiuso in $\hi \times \hi$.
\end{definition}

È imporante notare che se $\hi$ è spazio di Hilber infinito dimensionale allora non è detto che $\overline{\mathcal{G}(T)}$ sia il grafico di qualche operatore lineare.

\begin{definition}
    Se $A$ è un operatore in $\hi$ diremo che $B$ è \emph{estensione} di $A$ se $\mathcal{G}(A) \subseteq \mathcal{G}(B) $ e scriveremo che $A \subseteq B$. 
\end{definition}

\begin{example}
    Possiamo costruire una relazione d'ordine siano infatti $T_1, T_2$ operatori in $\hi$ diremo che $T_1 \leq T_2$ se $\mathcal{D}(T_1) \subseteq \mathcal{D}(T_2)$ e per ogni $\psi \in \mathcal{D}(T_1)$ vale $T_1 \psi = T_2 \psi$.

    Nell pratica considerando $\hi = L^2(\mathbb{R})$ e $T_1 = -i \frac{d}{dx}$ e $T_2 = -i \frac{d}{dx}$ con $\mathcal{D}(T_1) = C^\infty_c(\mathbb{R})$ e $\mathcal{D}(T_2) = H^1(\mathbb{R})$ allora $T_1 \leq T_2$.
\end{example}

\begin{definition}
    Diremo che un operatore $T$ è chiudibile se esiste un operatore $B$ tale che $T \subseteq B$ e $B$ è chiuso. Allora la minima estensione chiusa di $T$ detta chiusura $\overline{T}$ è tale che $\mathcal{G}(\overline{T}) = \overline{\mathcal{G}(T)}$.
\end{definition}

\begin{example}
    Dato $T_1= -i \dfrac{d}{dx}$ con $\mathcal{D}(T) = C^1_c(\R)$ è possibile mostrare che $T_1$ non è chiuso ma è chiudibile con chiusura $T_2 = -i \dfrac{d}{dx}$ in cui la derivata è in senso distribuzionale su $\mathcal{D}(T_1)=H^1(\R)$.
\end{example} 


\begin{proposition}
    Sia $T$ operatore nello spazio di Hilbert $\hi$. I seguenti fatti sono equivalenti:
\begin{enumerate}
    \item $T$ è chiudibile.
    \item $\overline{\mathcal{G}(T)}$ è il grafico di un operatore lineare.
    \item $\overline{\mathcal{G}(T)}$ non contiene elementi del tipo $(0,z)$ con $z \neq 0$.
\end{enumerate}
\end{proposition}


\begin{theorem}
    Tra le seguenti proprietà due implicano la terza:
\begin{enumerate}
    \item $T$ è chiuso.
    \item $T$ è limitato.
    \item $\mathcal{D}(T)=\hi$ ossia $T$ è ovunque definito.
\end{enumerate}
\end{theorem}

Sia $T \in \B$ un operatore limitato su $\hi$ per $\psi \in \hi$ fissato considero il funzionale lineare $\phi \mapsto \langle \psi, T \phi \rangle$ per $\phi \in \hi$. Per il teorema di rappresentazione di Riesz esiste un unico vettore $T^* \psi \in \hi$ tale che $\langle \psi, T \phi \rangle = \langle T^* \psi, \phi \rangle$ per ogni $\phi \in \hi$.

\begin{definition}
    Sia $T \in \B$ un operatore limitato su $\hi$. Definiamo \emph{l'aggiunto} di $T$ come l'operatore $T^*$ tale che per ogni $x, y \in \hi$ vale $\langle  x, Ty \rangle = \langle T^* x,  y \rangle$.
\end{definition}

Possiamo generalizzare la definizione di aggiunto per operatori illimitati. Sia $T$ un operatore illimitato su $\hi$ definito su un dominio denso $\mathcal{D}(T) \subseteq \hi$. Fisso $psi \in \hi$ considero il funzionale lineare non limitato $f_\psi : \mathcal{D}(T) \to \mathbb{C}$ definito da $f_\psi(\phi) = \langle \psi, T \phi \rangle$. Posso però considerare $\{ \psi \in \hi | f_\psi \text{ è limitato} \} = \{ \psi \in \hi | \sup_{\phi \in \mathcal{D}(T)} \frac{|\langle \psi, T \phi \rangle|}{\| \phi \|} < \infty \} = \mathcal{D}(T^*)$. 
Posso quindi utilizzare il teorema di rappresentazione di Riesz su $\mathcal{D}(T^*)$ per definire l'aggiunto di $T$.

\begin{definition}
    Sia $T$ un operatore illimitato su $\hi$ con dominio denso $\mathcal{D}(T)$. Definiamo \emph{l'aggiunto} di $T$ come l'operatore $T^*$ tale che per ogni $\psi \in \mathcal{D}(T^*)$ e $\phi \in \mathcal{D}(T)$ vale $\langle  \psi, T\phi \rangle = \langle T^* \psi,  \phi \rangle$.
\end{definition}

Ora dalla definizione non sempre avremo che $\mathcal{D}(T^*)$ è denso in $\hi$ per cui non avrà sempre senso parlare di $(T^*)^*$. Ma vale il seguente teorema:

\begin{theorem}
    Sia $T$ un operatore densamente definito su $\hi$ allora: 
\begin{enumerate}
    \item $T^*$ è chiuso.
    \item $T$ è chiudibile se e solo se $T^*$ è densamente definito. In tal caso $\overline{T} = T^{**}$.
    \item Se $T$ è chiudibile allora $(\overline{T})^* = T^*$.
\end{enumerate}
\end{theorem}

\begin{definition}
    Diremo che un operatore $T$ è \emph{autoaggiunto nel senso di Von Neumann} se $T = T^*$ e $\mathcal{D}(T) = \mathcal{D}(T^*)$.
\end{definition}

\begin{observation}
L'aggiunzione ribalta le inclusioni: se $T \subseteq S$, allora $S^* \subseteq T^*$. Sia ad esempio $T_i = -i \frac{d}{dx}$ e $\mathcal{D}(T) = C_c^\infty(\mathbb{R})\subseteq \mathcal{D}(T_2) = C_c^1(\mathbb{R} ) \subseteq \dots $, allora $\dots \subseteq \mathcal{D}(T^*_2) \subseteq \mathcal{D}(T^*_1)$, puó esistere quindi un "dominio di equilibrio" tale che $\mathcal{D}(T_{opt})=\mathcal{D}(T^*_{opt})$, in questo esempio $\mathcal{D}(T_{opt}) = H^1(\mathbb{R})$.
\end{observation}   

\textbf{Operatore posizione}: Tornando all'esempio dell'operatore posizione $(X_m\psi)(x)= x_m\psi(x)$ su $\hi = \Lr$ con $\mathcal{D}(X_m) = \Sw$ è simmetrico ma non autoaggiunto, il dominio infatti non è massimale, possiamo provare quindi ad estenderlo ad un dominio piú grande $\mathcal{D}(X_m):= \{ \psi \in \Lr | \int_{\R} |x|^2 |\psi(x)|^2 dx < \infty \}$, con questo dominio l'operatore è autoaggiunto. Vale la pena notare che questo procedimento non sempre è funzionale e non sempre si ottiene un operatore autoaggiunto, esistono peró due criteri di esistenza di estensioni autoaggiunte: il criterio di Von Neumann e il criterio di Nelson, l'operatore posizione in questo caso soddisfa il primo. Piú in generale sia $f : \R^d \to \mathbb{C}$ una funzione misurabile allora l'operatore moltiplicazione per $f$ definito da $(M_f \psi)(x) = f(x)\psi(x)$ è autoaggiunto se pongo $\mathcal{D}(M_f):=\{ \psi \in \Lr | \int_{\R^d} |f(x)|^2 |\psi(x)|^2 dx < \infty \}$. 


\section{Teoria spettrale su $\B$}

IL teorma spettrale per operatori finito dimensionali si genealizza al caso di operatori autoaggiunti in tre modi equivalenti:

\textbf{Spettralizzazione}: 
\begin{theorem}
    Sia $\hi$ spazio di Hilbert complesso separabile $A: \mathcal{D}(A) \to \hi$ autoaggiunto (non necessariamente limitato). Allora esistono:
\begin{enumerate}
    \item Uno spazio misurabile $(X, \mu)$.
    \item Una funzione $\lambda : X \to \mathbb{C}$ $\mu$-misurabile.
    \item Una mappa unitaria $U:\hi \to L^2(X,d \mu)$ tale che $(UAU^{-1} \psi)(x) = \lambda(x) \psi(x)$ per ogni $\psi \in \mathcal{D}(U):= \{ \psi \in L^2(X,d\mu)| \; \lambda \psi \in L^2(X, d \mu)\}$ o equivalentemente che $UAU^{-1} = M_\lambda$.
\end{enumerate}
\end{theorem}

Nota che in questo caso si è scelto $\lambda: X \to \mathbb{C}$ e non $\lambda: X \to \mathbb{R}$, questo perché in generale non è detto che $\lambda$ sia reale, valendo il teorema anche per $A$ unitari e antiautoaggiunti.

\begin{example}
    \begin{enumerate}
        \item %da fare
    \end{enumerate}
\end{example}

\textbf{Calcolo funzionale}: 
\begin{theorem}
    Sia $A$ operatore autoaggiunto su $\hi$. Esiste unica la mappa $\hat{\Phi}_A: M_b(\R) \to \B$ tale che:
\begin{enumerate}
    \item $\hat{\Phi}_A$ è uno $*$-omomorfismo di $*$-algebra 
    \item $\hat{\Phi}_A$ è continuo rispetto alla norma $\|
    \hat{\Phi}_A(f) \|_{\B} \leq \| f \|_\infty$
    \item $A$ è $\hat{\Phi}_A$-approssimabile in senso debole: Sia $h_n(x) \to x$ as $n \to \infty$ then for any $\psi \in \mathcal{D}(A)$ we have $\lim_{n \to \infty} \hat{\Phi}_A(h_n) \psi \to A \psi $
    \item Se $h_n(x) \to h(x)$ puntualmente allora $\hat{\Phi}_A(h_n) \to \hat{\Phi}_A(h)$ in topologia forte.
    \item Se $\psi$ è autovettore di $A$ e vale $A \psi = \lambda \psi$ allora $\hat{\Phi}_A(h)\psi = h(\lambda)\psi$ 
    \item Se $h$ è una funzione positiva allora $\hat{\Phi}_A(h)$ è positivo in senso operatoriale.
\end{enumerate}
\label{thm:funcal}
\end{theorem}

\begin{corollary}
    Sia $A$ operatore autoaggiunto su $\hi$ e $f_t(s)= e^{-its}$ allora $\hat{\Phi}_A(f_t) = U(t) := e^{-itA}$ è uniatrio e valre $U(t)U(s) = U(t+s)$.
\end{corollary}

\begin{corollary}
    Sia $\Omega \subseteq \R$ misurabile $f(s) := \chi_\Omega(s)$ allora $\hat{\Phi}_A(f) = E_\Omega$ è un proiettore ortogonale.
\end{corollary}

Le dimostrazioni dei corollari vengono dirette data la definizione di $*$-omomorfismo.

\textbf{Integrazione rispetto a PVM}

\begin{definition}
    Sia $\hi$ uno spazio di Hilbert complesso separabile, $X$ uno spazio topologico $\mathscr{B}(X)$ la $\sigma$-algebra di Borel su $X$. Un'applicazione $P: \mathscr{B}(X)\to \B$ è detta \emph{misura a valori di proiezione} (\emph{PVM}) su $X$ se per ogni $E,E'\in \B$ valgono:
\begin{enumerate}
    \item $P(E)$ è un proiettore ortogonale.
    \item $P({\varnothing})=0$ e $P(X)=\un$.
    \item $P(E)P(E')=P({E \cap E'})$.
    \item Sia $\{ E_n\} $ tale che $E_n \cap E_m = \varnothing$ quando $n \neq m$ e $\cup E_n = E$ allora vale $s-\sum P(E_n)=P(E)$
\end{enumerate}
\end{definition}

È possibile dimostrare che $P^A(E):=\hat{\Phi}_A(\chi_E)$ è una misura a valori di proiezione chiamata misura proiettiva associata ad $A$.

Ora con questa nozione di misura vogliamo provare ad integrare. 
\begin{theorem}
    Sia $P^A: \mathscr{B}(X)\to \B$ la PVM di cui sopra, $X$ uno spazio topologico a base numerabile e $\hi$ uno spazio di Hilbert, fissati $\psi, \phi \in \hi$ si consideri l'applicazione $$\mu_{\psi,\phi}^A: \mathscr{B}(X) \ni E \mapsto \langle \psi , \hat{\Phi}_A(\chi_E) \phi\rangle$$ è una misura complessa chiamata \emph{misura spettrale complessa associata a $\psi$ e $\phi$ }. Se $\psi = \phi$ allora $\mu_\psi^A := \mu_{\psi,\psi}^A$ è una misura positiva finita su $X$, inoltre per la definizione di PVM che abbiamo dato $\mu_{\psi, \phi}^A(X)=\langle \psi, \phi\rangle$ e in particolare $\mu_\psi^A(X)=\| \psi\|^2$ quindi preso $\|\psi\|=1$ avremo che $\mu_\psi^A$ è una misura di probabilità. 
\end{theorem}

Grazie al precedente teorema possiamo dare un senso concreto all'integrazione rispetto ad una PVM.
\begin{proposition}
    Sia $Q$ una forma quadratica limitata sullo spazio di Hilbert $\hi$ allora esiste unico $A \in \B$ tale che $Q(\psi)= \langle \psi , A \psi \rangle$ per ogni $\psi \in \hi$, se inoltre $Q(\psi) \in \R$ per ogni $\psi \in \hi$ $A$ è un operatore autoaggiunto. 
\end{proposition}
\begin{proof}* 
    Poichè $Q$ è limitata anche la forma sisquilineare $L$ associata a $Q$ è limitata, possiamo applicare il teorema di Riesz al funzionale $\phi \mapsto L(\psi, \phi)$ per cui esiste unico $\chi \in \hi$ tale che $L(\psi, \phi) = \langle \chi , \phi \rangle$. Definendo ora $B\psi := \chi$ e prendendone l'aggiunto avremo l'operatore cercato. Se inoltre $Q$ reale avremo che $\langle \psi |A\phi \rangle =L(\psi, \phi)=\overline{L(\psi, \phi)}= \overline{\langle \psi |A \phi \rangle} = \langle A \psi | \phi \rangle$
\end{proof}

Sia $f: X \to \mathbf{C}$ limitata e misurabile, consideriamo ora la mappa $Q_f : \hi \to \mathbf{C}$ definita :
$$Q_f(\psi)= \int_{X}f(x) d \mu_\psi^A(x)$$
Per prima cosa notiamo che $Q_f(\psi) \leq \| f\|_\infty \| \psi \|^2$ per le osservazioni sopra, quindi per la precedente proposizione esiste $B$ tale che $Q_f(\psi) =\langle \psi , B \psi \rangle$ è possibile dimostrare che $B$ soddisfa le prime 4 ipotesi del teorema \ref{thm:funcal} per cui vale $B = \hat{\Phi}_A(f)$. Introdurremo ora la seguente notazione $$ f(A)= \int_X f(\lambda)dE(\lambda)$$ Ora con questa nuova notazione possiamo generalizzare le due versioni precedenti del teorema spettrale, abbiamo visto che se $f \in M_b(X)$ allora ha per dominio tutto $\hi$ per generalizzare al caso in cui $g \in M(X)$ dobbiamo per prima cosa definire un dominio per l'operatore $g(A)$ per avere un integrale ben definito $$\mathcal{D}(g(A)):= \{ \psi \in \hi | \overline{g}g\in L^1(X , d\mu_\psi^A)\}$$ a questo punto possiamo definire in maniera naturale $$\langle \psi g(A)\phi \rangle = \int_X g(\lambda) d\mu_\psi(\lambda)$$ Possiamo ora enunciare il teorema spettrale per le PVM
\begin{theorem}
    Sia $A$ operatore autoaggiunto sullo spazio di Hilbert $\hi$ (non necessariamente limitato) allora esiste unica PVM $P^A: \mathscr{B}(X)\to \B$ tale che $$A= \int_{\R} \lambda dP^A(\lambda)$$ 
\end{theorem}

\section{Simmetrie quantistiche}

\begin{theorem}
    Sia $U(t)=e^{-itA}$ definito tramite il calcolo funzionale. Allora: 
\begin{enumerate}
    \item $\{ U(t)\}$ è un gruppo ad un parametro unitario, in particolare $U(0)=\un$
    \item $U$ è fortemente continuo, ossia per ogni $\psi \in \hi$ e per ogni $t_0 \in \R$ vale $\lim_{t \to t_0}(U(t)- U(t_0))=0$ 
    \item Per ogni $\psi \in \mathcal{D}(A)$ si ha che esiste $\lim_{\epsilon \to 0} \dfrac{1}{\epsilon}(U(\epsilon)- \un)\psi=-i A \psi$
    \item Se esiste il limite $\lim_{\epsilon \to 0} \dfrac{1}{\epsilon}(U(\epsilon)- \un)\psi$ allora $\psi \in \mathcal{D}(A)$
\end{enumerate}
\end{theorem}