\chapter{Monaco}

\begin{definition}
    Sia $T$ operatore nello spazio di Hilbert $\hi$:
\begin{enumerate}
    \item Chiameremo insieme risolvente di $T$ l'insieme $\rho(T)$ dei numeri complessi $z$ tali che:
    \begin{itemize}
        \item $T-z\un$ è iniettivo.
        \item $\Ran(T- z \un)$ è denso in $\hi$.
        \item $(T-z\un)^{-1}$ è limitato.
    \end{itemize}
    \item Chiameremo risolvente di $T$ l'operatore $(T-z\un)^{-1}$ per $z \in \rho(T)$.
    \item Chiameremo spettro di $T$ l'insieme $\sigma(T) := \mathbb{C} \setminus \rho(T)$. Lo spettro è unione disgiunta di tre insiemi:
    \begin{itemize}
        \item Spettro puntuale $\sigma_p(T)$ formato dai $\lambda \in \mathbb{C}$ tali che $T-\lambda \un$ non è iniettivo.
        \item Spettro continuo $\sigma_c(T)$ formato dai $\lambda \in \mathbb{C}$ tali che $T-\lambda \un$ è iniettivo, $\Ran(T-\lambda \un)$ è denso in $\hi$ ma $(T-\lambda \un)^{-1}$ non è limitato.
        \item Spettro residuale $\sigma_c(T)$ formato dai $\lambda \in \mathbb{C}$ tali che $T-\lambda \un$ è iniettivo e $\Ran(T-\lambda \un)$ non è denso in $\hi$.
    \end{itemize}
\end{enumerate}
\end{definition}


\section{Sistemi invarianti per rotazioni}


\section*{Modellizzazione matematica dell'atomo di idrogeno}
Studieremo la dinamica di una particella quantistica carica $e^-$ sotto l'azione del campo di forze elettrostatiche generato dal nucleo: 

\begin{equation*}
-\nabla V(\mathbf{r}) = \mathbf{F}(\mathbf{r}) = \dfrac{\gamma}{|\mathbf{r}|^3}\mathbf{r} , \quad V(\mathbf{r}) = \frac{-e^2}{|\mathbf{r}|}
\end{equation*}

Il modello quantistico ha per hamiltoniana:
\begin{equation*}
H = H_0 + V = \frac{\mathbf{p}^2}{2m} + V(\mathbf{r})
\end{equation*}
Più in generale, $H = H_0 + V$, che sono sistemi invarianti per rotazioni quindi 
\begin{equation*}
G = SO(3) = \{R \in M_{3 \times 3}(\mathbb{R}) : R^T R = I, \det R = 1\}
\end{equation*}
è il gruppo di simmetrie dinamiche del sistema.

Per descrivere $SO(3)$ possiamo specificare:
\begin{itemize}
    \item Asse di rotazione $\hat{n} \in S^2$;
    \item Angolo di rotazione $\alpha \in [\pi, \pi]$.
\end{itemize}

Prendere un tale $\alpha$ in tale dominio è ridondate poichè $R(\alpha, \hat{n}) = R(-\alpha, -\hat{n})$ ma tale scelta ci sarà utile in seguito.

Esempio:

\begin{equation*}
    R(\alpha, \hat{e_3}) = \begin{pmatrix}
    \cos \alpha & \sin \alpha & 0 \\           
    -\sin \alpha & \cos \alpha & 0 \\
    0 & 0 & 1   
\end{pmatrix}
\end{equation*}

Per questa lezione lavoreremo nel costesto $\hi = L^2(\mathbb{R}^3)$ facendo agire $SO(3)$ sul sistema quantistico come:
\begin{equation*}
   SO(3) \to \mathcal{U}(\hi) \quad R \mapsto (U_R \psi)(\mathbf{r}) := \psi(R^{-1}\mathbf{r})
\end{equation*}

Diciamo che $H$ descrive un sistema quantistico invariante per rotazioni se vale $U^{-1}_R H U_R = H$ per ogni $R \in SO(3)$ dal teorema di Noether $U^{-1}_R H U_R = H$ se e solo se $[H, L_i] = 0$ dove $L_i$ sono i generatori infinitesimali di $SO(3)$.

Da Stone:
\begin{equation*}
U(\alpha, \hat{e_i}) = e^{\frac{-i\alpha L_i}{\hbar} }, \quad L_i=L_i^*, \mathbf{L}=(L_1, L_2, L_3)
\end{equation*}
Dove $L_1, L_2, L_3$ sono gli operatori momento angolare, infatti 

\begin{proposition}
    $\mathbf{L}$ sono gli operatori momento angolare. Quindi $\mathbf{L}= \mathbf{x} \times \mathbf{p}$.
\end{proposition}

\begin{proof}
    Per brevità dimostreremo che $L_3 = x_1 p_2 - x_2 p_1 = -i \hbar (x_1 \partial_2 - x_2 \partial_1)$: %mettere a posto gli indici ora non mi va.
\end{proof}

\begin{proposition}
Gli operatori momento angolare soddisfano le regole di commutazione dell'algebra di Lie $\mathfrak{so}(3)$:
\begin{align*}
[L_1, L_2] &= i\hbar L_3, \\
[L_3, L_1] &= i\hbar L_2, \\
[L_2, L_3] &= i\hbar L_1.
\end{align*}
\end{proposition}
\begin{proof}
    Anche qui per brevità dimostreremo solo il primo commutatore:
\begin{align*}
    [L_1, L_2] =& [x_2 p_3 - x_3 p_2, x_3 p_1 - x_1 p_3] =\\
=& [x_2p_3, x_3p_1] - [x_2p_3, x_1p_3] - [x_3p_2, x_3p_1] + [x_3p_2, x_1p_3] =\\
=& i\hbar(x_1p_2 - x_2p_1) = i\hbar L_3
\end{align*} 
\end{proof}

Definiamo l'operatore di Casimir:
\begin{equation*}
\mathbf{L}^2 = L_x^2 + L_y^2 + L_z^2
\end{equation*}

\begin{proposition}
    L'operatore di Casimir commuta con ogni componente di $L$:
\begin{equation*}
[\mathbf{L}^2, L_i] = 0 \quad \forall i \in \{x, y, z\}
\end{equation*}
\end{proposition}

\begin{proof}
    Dimostriamo che $[\mathbf{L}^2, L_3] = 0$:

 \begin{align*}
    [\mathbf{L}^2, L_3] =& [L_1^2 + L_2^2 + L_3^2, L_3] = \\
  =& [L_1^2, L_3] + [L_2^2, L_3]= L_1[L_1, L_3] + [L_1, L_3]L_1 + L_2[L_2, L_3] + [L_2, L_3]L_2 = \\ 
=&  i\hbar L_1L_2 - i\hbar L_2L_1 - i\hbar L_1L_2 + i\hbar L_2L_1 = 0
\end{align*}
\end{proof}

Passiamo ora in coordinate sferiche per sfruttare la simmetria del sistema:
\begin{lemma}
    Esiste un isomorfismo unitario: 
\begin{align*}
\mathcal{R} :   L^2(\mathbb{R}_+) \Ht L^2(\mathbb{S}^2) \to& L^2(\mathbb{R}^3) \\ 
(r, \theta, \phi) \mapsto& f(r,\theta, \phi) := \dfrac{1}{r}R(r)Y(\theta, \phi)
\end{align*}
\end{lemma}

\begin{proof}
    \begin{align*}
    \|f\|^2=& \int_{\mathbb{R}^3} |f(\mathbf{x})|^2 \dd^3 x = \\
 =& \int_{\mathbb{R}_+ \times \mathbb{S}^2}r^2sin \theta \left| \frac{1}{r}R(r) Y (\theta, \phi)\right|^2 \dd r \dd \theta \dd \phi =\\ 
 =&  \int_{\mathbb{R}_+} |R(r)|^2 \dd r \int_{\mathbb{S}^2} |Y(\theta, \phi)|^2 \dd \theta \dd \phi =: \|R \otimes Y\|^2
    \end{align*}
 $f$ allora si estende ad un isomorfismo unitario. 
\end{proof}

%qui manca un pezzettino 

Ora per coloro che studiano matematica piccola digressione sul motivo formale per cui è interessante studiare l'operatore di Casimir, è immediato vedere come l'operatore di Casimir cosí definito sia un elemento del centro dell'algebra inviluppante universale di $\mathfrak{so}(3)\simeq \mathfrak{sl}(2)$ e quindi agirà come moltiplicazione per uno scalare sui moduli di Verma associati, ora è possibile dimostrare che è anche l'unico elemento del centro (per cui definisce univocamente il carattere associato al modulo di Verma) per cui per il teorema di Harish-Chandra basterà studiare la sua azione per conoscere completamente il modulo irriducibile associato. Per maggiori informazioni sulle rappresentazioni di $\mathfrak{so}(3)$ (che espande leggermente quello che vedremo tra poco) si veda \cite{Humphreys} Sezione 7, per una esaustiva trattaziona del teorema di Harish-Chandra, che esula dagli argomentri trattati in questo corso, si veda sempre \cite{Humphreys} Sezione 23. 


Per un sistema quantistico con simmetria rotazionale, lo spettro dell'operatore di Casimir è discreto:
\begin{equation*}
\mathbf{L}^2 Y_{\ell m} = \hbar^2 \ell(\ell+1) Y_{\ell m}
\end{equation*}
dove $\ell \in \mathbb{N}$ e $m \in \{-\ell, \dots, \ell\}$.

\section*{Conclusione}
La simmetria del gruppo $SO(3)$ permette di caratterizzare le proprietà dell'atomo di idrogeno e degli operatori associati.


\begin{definition}
    Un operatore Hermitiano $V$ è detto $H_0$-limitato se $\mathcal{D}(V) \supset \mathcal{D}(H_0)$ e esistono $C,D> 0$ tali che $\| V \psi \| \leq C \| H_0 \psi \| + D \| \psi \|$ per ogni $\psi \in \mathcal{D}(H_0)$. 
\end{definition}

È possibile dimostrare che $V$ è $H_0$-limitato se e solo se $\mathcal{D}(V) \subset \mathcal{D}(H_0)$ e $V(H_0 -z \un)^{-1}\in \B$ per ogni $z \in \rho(H_0)$. Definiamo inoltre il limite di $V$ rispetto a $H_0$ come $H_0\lim(V) = \sup_{z \in \rho(H_0)} \| V(H_0 -z \un)^{-1}\|$.

\begin{theorem}[Katô-Rellich]
    Sia $V$ un operatore hermitiano $H_0$-limitato con $H_0\lim(V)< 1$ allora $H = H_0 + V$ è autoaggiunto su $\mathcal{D}(H_0)$.
\end{theorem}

\begin{definition}
    Un operatore Hermitiano $V$ è detto $H_0$-compatto se $\mathcal{D}(V) \supset \mathcal{D}(H_0)$ e $V(H_0 -z \un)^{-1}\in \Bo$.
\end{definition}

Se $V$ è $H_0$-compatto allora $V$ è $H_0$-limitato e $H_0\lim(V) = 0$, inoltre $H_0 + V$ è essenzialmente autoaggiunto.

\begin{theorem}[Weyl]
    Perturbazioni $V$ $H_0$-compatte di un operatore autoaggiunto $H_0$ sono tali che $\sigma_{ess}(H_0 + V) = \sigma_{ess}(H_0)$.
\end{theorem}

\section{Stati puri e stati misti}
Chiameremo $\Bo$ l'insieme degli operatori compatti su $\hi$.

\begin{definition}
    Sia $\hi$ con la norma indotta $\| \dot \|$ dal prodotto scalare, diremo che $A \in \B$ è un operatore di Hilber-Schmidt se esiste una base hilbertiana $\{ u_k\}$ tale che $\sum \|Au_k \|<\infty$. 
\end{definition}

Indicheremo la classe di operatori di Hilber-Schmidt su $\hi$ come $\Hs$, è possibile inoltre dimostrare che $\Hs \subset \Bo$ è uno *-ideale bilatero.

\begin{proposition}
    Sia $\hi$ spazio di Hilbert, $T\in \B$. I seguenti tre fatti sono equivalenti:
\begin{enumerate}
    \item Esiste $N$ base hilbertiana tale che $\{ \langle u, |T| u \rangle \}$ ha somma finita.
    \item $\sqrt{T^*T}$ è di Hilbert-Schmidt.
    \item $T$ è compatto e la successione degli autovalori $\{m_n\}$ di $\sqrt{T^*T}$ contati con molteplicità ha somma finita. 
\end{enumerate}
\end{proposition}

\begin{definition}
    Sia $\hi$ spazio di Hilbert, $T\in \B$ è detto operatore classe traccia se vale una delle tre condizioni equivalenti della proposizione precedente. L'insieme degli operatori classe traccia su $\hi$ sarà indicato con $\Ct$, infine se $T \in \Ct$ allora definiamo $\| T\|_1 = \| \sqrt{T^*T}\|_2^2= \sum m_n$.
\end{definition}

Come per gli operatori di Hilbert-Schmidt $\Ct$ è uno *-ideale bilatero di $\B$. Inoltre $\Ct \subset \Hs \subset \Bo \subset \B$. Infine dato $A\in \Ct$ esistono $B, C \in \Hs$ tali che $A = BC$, viceversa se $B,C \in Hs$ allora $BC \in \Ct$.  
\begin{definition}
    Sia $\hi$ spazio di Hilbert, $T\in \Ct$, il numero $\mathbb{C} \ni \Tr T = \sum \langle u, Tu \rangle$ è detto traccia di $T$.
\end{definition}

Ora enunceremo un teorma con una serie di proprietà delle tracce, che non dimostreremo per brevità ma che si possono trovare in \cite{Mor}.

\begin{theorem}
    Sia $\hi$ spazio di Hilbert, $T \in \Ct$ allora valgono le seguenti:
\begin{itemize}
    \item $|T| \in \Ct$ e $\Tr |T| = \sum |m_n|=\|T\|_1$.
    \item Se $S \in \B$ allora $\Tr(ST) = \Tr(TS)$.
    \item Se $\hi$ è complesso allora $\Tr(T^*) = \overline{\Tr(T)}$.
    \item Se $T \in \Ct$ e $T = \sum T_i$ con $T_i \in \Ct$ allora $\Tr(T) = \sum \Tr(T_i)$.
    \item Se $\hi$ è complesso allora $\Tr(T) = \sum_{\lambda \in \sigma_p(T)} \lambda$, dove i $\lambda$ sono contati con molteplicità geometrica. 
\end{itemize}
\end{theorem}

La dimostrazione dell'ultimo punto è interessante ma piuttosto articolata e puó essere trovata in \cite{Bir}. 


\begin{definition}
    Una mappa lineare $\mathbb{E}: \B \to \mathbb{C}$ è detta \emph{ una famiglia di valori attesi} se valgono le seguenti:

\begin{itemize}
    \item $\mathbb{E}(\un) = 1$.
    \item $\mathbb{E}(A)$ è reale quando $A$ è autoaggiunto.
    \item $\mathbb{E}(A)$ è positivo quando $A$ è autoaggiunto e positivo.
    \item Per ogni successione $A_n \in \B$ se $\| A_n\psi - A \psi\| \to 0$ per tutti $\psi \in \hi$ allora $\Phi(A_n) \to \Phi(A)$.
\end{itemize}
\end{definition}

\begin{definition}
    Un operatore $\rho\in \Ct$ è una matrice densità se $\rho$ è autoaggiunto, non negativo e vale $\Tr \rho =1$.
\end{definition}

\begin{definition}
    Siano $\hi_1, \hi_2$ spazi di Hilber definiremo il \emph{prodotto tensore tra due spazi di Hilbert} $\hi_1 \hat{\otimes} \hi_2$ come il completamento di $\hi_1 \otimes \hi_2$ rispetto al prodotto:
$(u_1 \otimes v_1, u_2 \otimes v_2)= (u_1, u_2)_1 (v_1, v_2)_2$.
\end{definition}

È naturale definire il prodotto tensore tra operatori, siano $A_1 \in \mathfrak{B}(\hi_1)$ e $A_2 \in \mathfrak{B}(\hi_2)$ definiremo $A_1 \otimes A_2 \in \B(\hi_1 \hat{\otimes} \hi_2)$ come $A_1 \otimes A_2(u_1 \otimes u_2) = A_1 u_1 \otimes A_2 u_2$.
A questo punto sia $\rho$ matrice dentià su $\hi_1 \Ht \hi_2$ allora $\rho^{(1)}$ e $\rho^{(2)}$ sono le matrici densità ridotte su $\hi_1$ e $\hi_2$ rispettivamente definite come le uniche tali che $\tr(\rho(A \otimes \un))= \tr(\rho^{(1)}A)$ e $\tr(\rho(\un \otimes B))= \tr(\rho^{(2)}B)$ per ogni $A \in \mathfrak{B}(\hi_1)$ e $B \in \mathfrak{B}(\hi_2)$. Esistenza e unicità di tali matrici densità è dimostrata ad esempio in \cite{Hall} Theorem 19.13.


\begin{proposition}
    L'applicazione $ L^2(X_1, \mu_1) \Ht L^2(X_2, \mu_2) \to L^2(X_1 \times X_2, \mu_1 \times \mu_2) $ è un isomorfismo.
\end{proposition}

Per la dimostrazione di veda \cite{Hall}.

Se ho n sistemi quantistici composti descritti da $\hi_1, \dots, \hi_n$ allora il sistema composto è descritto dallo spazio di hilber $\hi = \hi_1 \Ht \dots \Ht \hi_n$. 
Per cui un sistema quantistico di n particelle puó essere descritto da $L^2(\mathbb{R}^{nd})= \Lr \Ht \dots \Ht \Lr$. Ora vogliamo generalizzare gli operatori a questi spazi tensore, 
siano $A_i^* = A_i\in \mathfrak{B}(\hi_1)$ definiremo l'osservabile del sistema composto come $A(\psi_1 \otimes \dots \otimes \psi_n)= A_1 \psi_1 \otimes \dots \otimes A_n \psi_n$.
In particolare:
\begin{enumerate}
    \item Se $A_i^* = A_i\in \mathfrak{B}(\hi_i)$ definisco $\B \ni A^{ (i)} := \un \otimes \un \otimes \dots \otimes \un \otimes A_i \otimes \un \otimes \dots \otimes \un$.
    \item Se $I = \{ i_1, \dots , i_k \} \subset \{1, \dots , n \}$ e $A \in \mathfrak{B}(\hi_{i_1}\Ht \dots \Ht \hi_{i_k})$ allora $A^{ (I)} \in \B$ agisce sui sottosistemi associati.
\end{enumerate}

\begin{theorem}
    Se $\rho^* = \rho \geq 0$ $\Tr(\rho) = 1$ matrice densità su $\hi$ allora siste una matrice densità $\rho^{(I)}$ su $\mathfrak{B}(\hi_{i_1}\Ht \dots \Ht \hi_{i_k})$ tale che $\mathbb{E}_\rho(A^{(I)}) = \mathbb{E}_{\rho^{(I)}}(A)$, con $\rho^{(I)}$ lo stato indotto da $\rho$ sul sottosistema $I$ chiamato matrice dentià ridotta e $\mathbb{E}_\rho^{(I)}$ è detta traccia parziale della famiglia $\mathbb{E}_\rho$.
\end{theorem}

\begin{proof}
    %Fare dopo
\end{proof}

